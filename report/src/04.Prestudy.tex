\chapter{Prestudy}

Before we started the actual development of GentleIDI, we needed to get
an overview of what options were available to us. There are many Web
development frameworks, and we were free to choose whichever we wanted.
In addition, we needed to get an understanding of the problems we were
intended to solve: what did the old system look like, what were its
major flaws?

\section{Learning Tools/Framework}

The first challenge we were faced with was that of deciding which
programming languages and frameworks we were to utilize. We were
recommended to use the Python-based framework called Django, both from
our customer and other developers. We decided to make a quick attempt
at getting comfortable with Python and Django, and then later on make a
decision on whether we were to use it or not. Most of our group had
previous experience with Web development using JSP(JavaServer Pages),
and JSP was our primary fallback in case we decided Django was not
suitable for our needs. 

We were able to master the basics of Django quite fast and its
advantages became apparent. It was quickly seen that Django had enough
features for the scope of our project, and a sizable community in case
we needed help to use the framework. In other words Django was chosen
quite early on, and mastering it became a priority during the initial
pre study phase. In order to make sure that everyone had a basic
understanding of the most central components of the framework, we
decided that every member of our group were to implement their own
Website providing basic functionality for posting and editing news
articles. The best article Website was chosen as the fundament for
GentleIDI, and is part of the end product. 

When we first started working on the execution nodes we needed to find
an appropriate way of sandboxing the user submissions. Running
untrusted code with no restrictions would be an unacceptable security
flaw, and we decided to put a lot of work into finding the best
possible way of securing our system. As a consequence we spent a lot of
work hours researching state-of-the-art security mechanisms in Linux,
including cgroups, AppArmor, and Vagrant for setting up virtual
machines. However, this effort proved unsuccessful and we ended up
using none of it. We ended up falling back to restricting access simply
by setting file permissions and creating dedicated users with reduced
privileges. 


Throughout the entire project we used Git for revision control, thus,
mastering Git became increasingly important as the complexity of the
code base increased. We never dedicated work hours to learning Git,
however, we kept a constant focus on getting better and using the
features available to us. By the end of the project we were all at a
level of competence where using Git saved us a lot of work and improved
our efficiency significantly. 


\section{Researching the Old System}

The old system consisted of a Python backend for evaluating submissions
and a PHP frontend. Though the system was in working order, a lot of
the management had to be done by source code modification, and direct
database manipulation. Initially we considered reusing parts of the old
system, but those plans were dismissed when we got access to its source
code. Hard coding, lack of modularity, and redundancy were in abundance
in the existing source. 


As a consequence we mostly referred to the source in order to understand
fine details of the system requirements. In addition to poor code
quality, some key design decisions crippled the old
system{\textquoteright}s scalability, such as using a SQLite database.
SQLite is mostly intended for mobile apps and development environments,
and unsuitable for large scale systems. Researching the old system made
us aware of several pitfalls, and for the rest of the project we kept a
strict focus on scalability and code quality. 

\section{Similar Software}

There are other systems that have functionality similar to our system.
Notable examples include Google Code Jam, NCPC (Nordic Collegiate
Programming Contest) and Facebook Hacker Cup. They are commonly
referred to as Competitive programming. There are some differences from
competition to competition, but the basis is the same. They are also
not open-source so it was impossible to get a look at the code.
Researching these similar systems never had any real impact on the
solution we decided to go for, and the research never lead to changes
in the requirements. This was mostly due to them being closed source
and the fact that our scope was quite large from the beginning.

\pagebreak
\section{Desired Solution}

The desired solution from the customers perspective was a system that
had the same functionality as the old system. But making it so that one
does not have to edit the source code, or the database directly, to
complete simple tasks. Except from the initial setup, all management
should be done through a simple Web interface available to the admins.
\ In addition, the customer wanted a scalable system, which could
handle massively large contests, given the hardware. 


We as students and developers had certain hopes as to what the end
product would look like. One of our greatest desires was that the end
product would be maintained and developed further in the years to come.
A great way of increasing the chances of that becoming a reality is
making sure that our product is open source. With the source available
to the public anyone can contribute to the project, making sure that it
continues to evolve. Another advantage for us is that with an open
source product anyone is free to inspect our work, and as such the
product becomes a great way for us to show off our abilities. 


Any development project with a scope as large as our{\textquoteright}s
is likely to reuse other people{\textquoteright}s code, incorporate
open source products etc. Making a product dependent on an open source
project that suddenly stops being maintained was a pitfall we
definitely wanted to avoid. Hence, our ideal end product would be based
on nothing but source that we were sure would be maintained for years
to come. With this in mind our ideal end product would be open source,
with code quality that we can feel proud sharing with others, and that
others would want to contribute to. 


\section{Study Result}

The prestudy lasted four weeks and our results regarding libraries and
development tools can be viewed in Project Management and Tools and
Frameworks. Development method chosen can also be viewed in Project
Management. For the technical choices see System Architecture. 

\section{Technology}

During the prestudy phase the group considered different frameworks to
be used. 

%
%Hva med � f� med l�ringsaspektet? En av de viktigere grunnene til at vi var skeptiske var at vi var redde for at det ville ta for lang tid � l�re.
%
%Derfor ogs� flask dukker opp
%
%django var foretrukket av kunde. stor pullfaktor
%Med unntak av flask s� er vel dette dekket av learning tools and framework?
\subsection{Django}

Django is a free and open source Web development framework, written
exclusively in Python. Everything from the framework itself, to the
applications intended to run on top of it, is implemented in Python,
even settings are written as Python scripts. Django is also based on
the MVC architecture, which suits us fine as we are all used to work in
an MVC context. It is structured in a way that emphasizes the DRY
principle. The combination of Python and Django lays a good foundation
for rapid development, and high maintainability. 


\pagebreak
\subsection{Flask}

An alternative to Django which we did consider was the Flask
micro-framework. Though Flask might have been a viable alternative, we
discovered it a little too late, about a week into our Django training.
Because we did not see any major advantages of Flask over Django, we
decided to stick to the latter. Like Django, Flask is also written in
Python. 


\subsection{JSP}

JSP stands for JavaServer Pages, and is a technology used to create
dynamic Web pages. As Django, it also adheres to the MVC approach. Some
of the group members had experience with JSP, but we decided not to use
it. The most commonly learned programming language at NTNU is Python.
Since JSP is based on Java we feared that the use of JSP would affect
the maintainability of the program. It was therefore not used.


\subsection{Backend Technology}

There are some security issues regarding letting the users run their
programs on our backend. In worst case a contestant could write harmful
code setting the contest to a halt. We discussed different approach to
eliminate this problem.

%
%Generelt var det ikke bare apparmor, men forskjellen p� en integrert l�sning vs hardkoding av regler
%
%Jeg ser ikke den store forskjellen. AppArmor hadde ogs� trengt definerte rules.
\subsection{AppArmor}

Abbreviation for Application Armor. It is a Linux kernel security module
allowing the system administrator to associate a security profile with
each program. This would restrict the capabilities of that program. We
think the optimal product would have used AppArmor, however we were
unable to make it work the way we wanted. One of the major issues we
had was that of restricting interpreted languages like Java and Python.
Programs written in these languages require an external binary to
execute them, and restricting this binary without breaking the rest of
the system was hard. E.g. in order to restrict submissions written in
Python we would end up restricting every single Python program running
on the system. Given more time we might have been able to make this
work, however, that was not the case. 




\subsection{Our Alternative to AppArmor}

The solution we finally decided to go for was to execute the submissions
as a sandboxed user. We created a user, removed its network access, and
made sure that the only relevant files it had access to was the
submission executables. The major weakness with this solution is that
security is dependent on file permissions per file. A single executable
with the wrong permissions could be a threat. There are some programs
that are capable of running subprocesses as a second user, and thereby
bypassing the network blocking. 


\section{Development Method}

At the beginning of our project we all had some experience with Scrum,
which made it the most obvious choice in terms of development method.
It is quite suitable for this type of small scale project and keeps the
planning overhead at a minimum. However, as far as we knew there could
be other, better suited methods for our project. The major requirements
we had to the development method was that it should be agile and
iterative. We also wanted to work in a test driven manner, and do most
of the programming in pairs. This made Extreme Programming a viable
alternative. 



Extreme Programming has a greater focus on pair programming and code
reviews, than Scrum. This was in tune with the way we wanted to work,
however, using XP would increase the amount of time needed to get
comfortable with the development method. There are some major pitfalls
related to using a development method you that you do not master, and
when it came down to making a decision we played it safe. We decided to
use Scrum and implement pair programming and test driven development in
our own way.


When the development started we soon realized that we did not have the
competence needed to do test driven development well, and this aspect
was dropped. We ended up using Scrum with a focus on working in pairs.
