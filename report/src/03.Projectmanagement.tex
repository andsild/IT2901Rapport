\section{Project Management}
This section will go through the different project roles we deemed
important. We will explain our development method, which tools we use
and give an overview of how we planned the project. Furthermore, in
section X.X we also provide a structured overview of how we organized
our time. 

\subsection{Project Roles}
We wanted to ensure that all developers had an even workload and
experience in all components of our project. To achieve this, we
maintained a flat organizational structure where all decisions were
made in groups, and no member would work alone on a task for a longer
period of time. Some tasks and delegations, however, would be easier to
assign only once to reduce time spent in transition between developers.
The following paragraphs discusses the different roles we assigned.

The most central role is that of the scrum master. The role mainly
consists of setting up meeting agendas and keeping control of what
team-members are working on. In addition, the scrum master should act
as a buffer between the team and other distractions. In our framework,
the scrum master also had a casting vote whenever there was a
disagreement. The group elected Haakon to be scrum master because of
his well-established authority and organization.

We also assigned the role of a transcriptionist. His job consists of
writing a short summary of every meeting, and making this available to
the rest of the group. This includes meetings with the customer and
supervisor. This job was performed by Anders, who volunteered for the
position. We randomly assigned Haakon to be customer contact, and
Tino as responsible for room reservations.

\subsection{Development Method (Scrum)}
Scrum focuses on having daily meetings, and constantly adjusting to
changes by iterative development. This makes it easier to predict and
to adjust for problems that may occur. It was hard to predict what
would happen in our project, therefore our sprints were short, lasting
at most two weeks. The transition between two sprints was done during a
prolonged meeting on Wednesdays. During this meeting we evaluated the
latest sprint and planned the upcoming one. Every team member were
requested to say three good things and three bad things regarding the
last sprint. This was followed by a discussion of how to plan the next
sprint better. Lastly we showed what had been completed, to the other
members of the group, before setting up the next sprint. Scrum also
focuses on having finished versions of the systems on each iteration,
and to finish all packages in the given iteration.
In order to take advantage of the best in everyone's
abilities we worked in pairs where this was efficient. Working in pairs
is common in agile development. This was to improve code quality and
reduce errors\footnote{[Page 2}
href{http://www.cs.pomona.edu/classes/cs121/supp/williams_prpgm.pdf}{http}

\subsection{Tools/Framework}
The customer wanted our end product to be easy to maintain for future
developer. Therefore we have chosen tools that are well known and easy
to learn. A lot of different tools were considered for this system. Some of the
most important are:

\begin{itemize}
    \item Django, a framework written in Python.
    \item Editors like VIM and Eclipse were used. 
    \item Documenting the development process was completed using google drive.
    \item Git was used as a version control system, with github as hosting
        service
    \item Communication were done through email lists, IRC and Facebook. 
    \item User interface design was stylized with bootstrap and grappelli. 
\end{itemize}
A full list of all tools and frameworks used can be viewed in appendix
\textit{Tools and Frameworks.}

\subsection{Project-Level Planning}
After our initial requirements elicitation we began to plan our
development process. The purpose of the plan was to verify that we had
enough time to complete the requirements, and to avoid unforeseen
risks. This section will present the various components we introduced
to structurize the project.

\subsection{Work Breakdown Structure}
WBS is a decomposition of the project into phases, deliverables and work
packages. Each package was further broken down into different tasks.
The benefits from doing these are as follows:
\begin{itemize}
    \item Planning out the entire process prevents bottlenecks.
    \item Clearly defining the scope o a package prevents excess or insufficient
        time usage. 
    \item It is easy for supervisors and other parties to evaluate and
        understand our process. 
\end{itemize}

Table X.X shows the work breakdown structure created. These high-level
packages were later broken down into activities, which are in the
product backlog, see appendix 
\begin{framed}
    \begin{enumerate}
        \item Project management
        \begin{enumerate}
            \item Write timesheet template
            \item Look at the reflection notes
            \item Meetings
            \begin{enumerate}
                \item Internal
                \item Customer
                \item supervisor
            \end{enumerate}
            \item  Report
            \begin{enumerate}
                \item Preliminary version
                \item Mid-semester version
                \item Final version
            \end{enumerate}
            \item Risk assessment
            \item WBS
            \item Status report
            \item Activity plans
        \end{enumerate}

        \item Pre-study        
        \begin{enumerate}
            \item Install and learn tools
            \item Learn language/framework
            \item Course
        \end{enumerate}
        \item Design
        \begin{enumerate}
            \item Requirement Specification
            \begin{enumerate}
                \item Functional
                \item Non-functional
            \end{enumerate}
            \item  System architecture
            \item Database modeling
            \item User Interface
            \begin{enumerate}
                \item Prototyping
                \item Usability Testing
            \end{enumerate}
            \item  Admin interface
        \end{enumerate}

        \item Development
        \begin{enumerate}
            \item  Backend
            \begin{enumerate}
                \item Execution-node(s)
                \begin{enumerate}
                    \item Web-page
                \end{enumerate}
            \begin{enumerate}
                \item User
                \item Usergroups
                \item Team management
            \end{enumerate}
            \item  Statistics
            \item Contest management 
            \item Clarification system
            \item Balloons system
            \item Unit testing
        \end{enumerate}

        \item Testing
        \begin{enumerate}
            \item  User-test
            \item System-test  
            \item Final test
        \end{enumerate}

        \item Implementation
        \begin{enumerate}
            \item  Deploy to production
            \item Installation
            \item Turn in to stakeholder
        \end{enumerate}

        \item Implementation
        \begin{enumerate}
            \item Verify
            \item Document
        \end{enumerate}
    \end{enumerate}
    \end{enumerate}
\end{framed}

We also created a gantt chart. Here, each package was assigned an
estimated time period, over how long time we expected to use. For ease
of comprehension, not every package was included from the WBS.\ The
gantt chart is shown in figure~\ref{gantt}
\newline
\begin{table}
\caption{Gantt chart}
\label{gantt}
\begin{tabular}{|l|l|l|l|l|l|l|l|l|l|l|l|l|l|l|l|}
\hline
\textbf{WP Name} & 1 & 2 & 3 & 4 & 5 & 6 & 7 & 8 & 9 & 10 & 11 & 12 & 13 & 14 & 15 \\
\hline
Project management &\rowcolor{ForestGreen}&&&&&&&&&&&&&& \\
\hline
WBS &&&&&&&&&&&&&&& \\
\hline
Pre-study
&\cellcolor{yellow}&\cellcolor{yellow}&\cellcolor{yellow}&\cellcolor{yellow}&&&&&&&&&&&
\\ \hline
Install and learn tools&\cellcolor{Dandelion}&\cellcolor{Dandelion}&\cellcolor{Dandelion}&\cellcolor{Dandelion}&&&&&&&&&&& \\
\hline
Learn language/framework&&\cellcolor{Dandelion}&\cellcolor{Dandelion}&\cellcolor{Dandelion}&&&&&&&&&&&  \\
\hline
Course &&\cellcolor{Dandelion}&\cellcolor{Dandelion}&&&&&&&&&&&&  \\
\hline
\textbf{Design} & &\cellcolor{MidnightBlue} &\cellcolor{MidnightBlue} &\cellcolor{MidnightBlue} &\cellcolor{MidnightBlue} &\cellcolor{MidnightBlue} &\cellcolor{MidnightBlue} &\cellcolor{MidnightBlue} &\cellcolor{MidnightBlue} &\cellcolor{MidnightBlue} & & & & & \\ \hline
Requirement specification&&\cellcolor{RoyalBlue}&\cellcolor{RoyalBlue}&\cellcolor{RoyalBlue}&\cellcolor{RoyalBlue}&\cellcolor{RoyalBlue}&\cellcolor{RoyalBlue}&\cellcolor{RoyalBlue}&\cellcolor{RoyalBlue}&\cellcolor{RoyalBlue}&&&&&  \\
\hline
System architecture&&&\cellcolor{RoyalBlue}&\cellcolor{RoyalBlue}&\cellcolor{RoyalBlue}&\cellcolor{RoyalBlue}&\cellcolor{RoyalBlue}&&&&&&&&  \\
\hline
Database modelling&&&\cellcolor{RoyalBlue}&\cellcolor{RoyalBlue}&\cellcolor{RoyalBlue}&\cellcolor{RoyalBlue}&&&&&&&&&  \\
\hline
Tests &&&\cellcolor{RoyalBlue}&\cellcolor{RoyalBlue}&\cellcolor{RoyalBlue}&\cellcolor{RoyalBlue}&&&&&&&&&  \\
\hline
User-interface&&&&&\cellcolor{MidnightBlue}&\cellcolor{MidnightBlue}&\cellcolor{MidnightBlue}&\cellcolor{MidnightBlue}&\cellcolor{MidnightBlue}&&&&&&  \\
\hline
\textbf{Development}&&&&\cellcolor{Purple}&\cellcolor{Purple}&\cellcolor{Purple}&\cellcolor{Purple}&\cellcolor{Purple}&\cellcolor{Purple}&\cellcolor{Purple}&\cellcolor{Purple}&&&&  \\
\hline
Execution node&&&&&&&&\cellcolor{Orchid}&\cellcolor{Orchid}&\cellcolor{Orchid}&\cellcolor{Orchid}&&&&  \\
\hline
Implement single node&&&&&&&&\cellcolor{Thistle}&\cellcolor{Thistle}&\cellcolor{Thistle}&&&&&  \\
\hline
Implement several nodes&&&&&&&&&\cellcolor{Thistle}&\cellcolor{Thistle}&&&&&  \\
\hline
Content Management System&&&&&&&&&&&\cellcolor{Orchid}&&&&  \\
\hline
Front end&&&&\cellcolor{Orchid}&\cellcolor{Orchid}&\cellcolor{Orchid}&\cellcolor{Orchid}&\cellcolor{Orchid}&\cellcolor{Orchid}&\cellcolor{Orchid}&\cellcolor{Orchid}&&&&  \\
\hline
\textbf{Testing}&&&&&&&\rowcolor{Red}&\rowcolor{Red}&\rowcolor{Red}&\rowcolor{Red}&\rowcolor{Red}&\rowcolor{Red}&\rowcolor{Red}&\rowcolor{Red}&\rowcolor{Red}  \\
\hline
Unit testing &&&&&&&\rowcolor{Orchid}&\rowcolor{Orchid}&\rowcolor{Orchid}&\rowcolor{Orchid}&\rowcolor{Orchid}&\rowcolor{Orchid}&\rowcolor{Orchid}&\rowcolor{Orchid}&\rowcolor{Orchid}  \\
\hline
Integration testing&&&&&&&&&\rowcolor{Melon}&\rowcolor{Melon}&\rowcolor{Melon}&\rowcolor{Melon}&\rowcolor{Melon}&\rowcolor{Melon}&\rowcolor{Melon}  \\
\hline
System test&&&&&&&&&\rowcolor{Melon}&\rowcolor{Melon}&\rowcolor{Melon}&\rowcolor{Melon}&\rowcolor{Melon}&\rowcolor{Melon}&\rowcolor{Melon}  \\
\hline
\textbf{Production}&&&&&&&&&&&&&&\rowcolor{MidnightBlue}&\rowcolor{MidnightBlue}  \\
\hline
\hline
Post-implementation&&&&&&&&&&&&&&&\rowcolor{Plum} \\
\hline
\end{tabular}
\end{table}

The gantt chart was revised several times during the first 4 sprints,
mainly due to new deadlines set by the customer. The original chart is
also given in appendix X.

\subsection{Milestones}
Throughout the project, the supervisor, customer, and the project group
set deadlines.
Some of the milestones marks the completion of work package(s). We have
four of these milestones, M-03, M-05, M-06 and M-07. The other
milestones represents events with deadlines that were given by the
course stakeholders. These are M-01, M-02, M-04, M-08. The group are
using the milestones in order to determine if the project is on
schedule and to monitor the progress.

\begin{description}
    \item[Preliminary report M-01]
    Preliminary report is the delivery of the first version of the report.
    This was intended to help us get started with important aspects of the
    project work.

    \item[Mid-semester report M-02]
    This version of the report should present all of the analysis and most
    of the design of our system. The delivery date for the mid-semester
    report is 16.03. We wanted to complete this earlier in order to focus
    on M-03.

    \item[First release M-03]
    This milestone marks the groups first delivery to the customer. The
    reader can view what functional requirements this release includes in
    the functional requirements. In summary this release should
    make it possible for contestants to sign up for a competition. Three
    days prior to the release the group will meet up with the customer and
    overlook that all the requirements are met. This meeting will also act
    as an introduction to the system, showing the customer how to manage
    the system. 

    \item[Presentation M.04]
    The main purpose of the presentation is for the class to share their
    experience with other groups. 

    \item[Beta-release M-05]
    The beta release should contain most of the major features, but it might
    not yet be complete. This version of the program should only be a
    release to a selected group of people. From M-06 to M-07 the system
    will be tested.

    \item[IDI Open test event M-06]
    On april the 26th we had a test event where everybody could test the
    system. This means that leading up to this event the system should be a
    release candidate. 

    \item[IDI Open M-07]
    This is the day of the competition and the system should be in a release
    version. 

    \item[Final report M-08]
    This milestone marks the final date for delivering the report as well as
    the final date of this bachelor thesis. Based on feedback received
    from the competition the group might choose to implement some changes
    to the system. 
\end{description}

\subsection{Meetings}
Throughout this project the group have had several meetings. They can be
categorized in three categories: internal, supervisor and customer
meeting. We established some meetings rules:
\begin{itemize}
    \item All meetings follow ``the academic quarter'', meaning that the time
        of start was XX.15.
    \item Members that were late had to bring a cake to the next meeting. 
    \item All members may at any time propose a coffee break. This proposal has
        to be followed. 
    \item No laptop should be open during the meetings. 
\end{itemize}
\subsubsection{Internal meetings}
We had three internal meetings each week. Two of which were daily scrum
meetings. These were primarily set to be on mondays and thursdays.
During these meetings each group member would answer three questions: 
\begin{itemize}
    \item What have you done since the last meeting?
    \item What are you planning to do until next meeting
    \item Do you have any problems regarding reaching your goal? 
\end{itemize}
The group would continue to work together after these meetings. 

On Wednesday we had longer meetings, marking the end of one sprint and
the beginning of the next. This meeting would consist of a sprint
review meeting and a sprint retrospective, where we discussed: 
\begin{itemize}
    \item What was good/bad with the last sprint
    \item What should we try to improve during the next sprint. 
\end{itemize}
After that we held a Sprint planning meeting and created a new sprint
backlog. Our official meetings structure for this meeting can be viewed
in the appendix. 

\subsubsection{Supervisor meeting}
Meetings with the supervisor was generally held at a bi weekly basis.
During these meetings we talked about what we had done, what we were
going to do and received feedback on what we had done. Before each
meeting we had to deliver status reports and activity diagrams. These
activity diagrams were early on replaced by sprint backlog and burndown
charts to facilitate the process. 

\subsubsection{Customer meeting}
Customer meetings were held whenever we felt that a certain part of the
requirements specification was unclear to us, and when we wanted
approval of a newly completed feature. Throughout the semester there
were a lot of meetings. As we never decided upon a fixed interval
between customer meetings, the frequency varied a lot. The couple of
days leading up to a release date often contained customer meetings in
order to get everything right before starting on the next release.
During our periods of focusing on writing this report, the frequency of
these meetings naturally went down as the product did not progress, and
as a consequence we had little to discuss with the customer. 

\subsection{Resources}
This section contains the available resources for the project. We
intended to use a minimum of 20/25 hours per person each week, but
prepared for more work as we approached the deadline. This estimate was
later scaled up to a minimum of 25/30 to weeks before easter, During
easter, the amount of hours per week scaled up higher.
Planned work 
Table X.X shows our first initial draft of sprints. 

\begin{tabular}{|l|l|l|l|}
\hline
 Sprint & Range (week) & Days &
 Hours\\\hline
 1 & 3 - 4 & 7 & 15\\\hline
 2 & 4 - 5 & 7 & 20\\\hline
 3 & 5 - 6 & 7 & 20 \\\hline
 4 & 6 - 7 & 7 & 20 \\\hline
 5 & 7 - 8 & 7 & 20\\\hline
 6 & 8 - 9  & 7 & 20\\\hline
 7 & 9 - 10 & 7 & 20\\\hline
 8 & 10 - 11 & 7 & 20\\\hline
 9 & 11 -12 & 7 & 20\\\hline
 10 & 12 -13 & 7 & 20\\\hline
 11 & 13 - 14 & 7 & 20\\\hline
 12  & 14 - 15  & 9 & 33\\\hline
 Easter & 15 - 17 & 12 & {}-\\\hline
 13 & 17 - 18 & 7 & 35\\\hline
 14 & 18 - 19 (Leading up to event) & 9 & 35\\\hline
 After & 19 - 22 & 21 & 35\\\hline
 Total:  & & 91 & 353\\\hline
\end{tabular}


\subsubsection{Actual work}
Table X.X shows the actual sprints and work done. The hours are for each
person, during that sprint.
\begin{tabular}{|l|l|l|l|}
\hline
 Sprint & Week & Days & Hours \\\hline
 1 & 3-4 & 7 & 15\\\hline
 2 & 4-5  & 7  & 15\\\hline
 3 & 5-6 & 7  & 20\\\hline
 4 & 6-7 & 7 & 20\\\hline
 5 & 7-8 (midterm report) & 7 & 27\\\hline
 6 & 8-9  & 7  & 31\\\hline
 7 & 10-11 & 7  & 35\\\hline
 8 & 11-12 & 7 & 30\\\hline
 9 & 12-13 & 7  & 30\\\hline
 10 & 14-15 & 9 & 40\\\hline
 11 & 15-17 \ (starting 16.04, ending 26.04, easter) & 10 & 90\\\hline
 12 & 18-19 & 6 & 35\\\hline
 After & 19.22 & 21 & 45\\\hline
 Total & & 100 & 433\\\hline
\end{tabular}
