\chapter{User stories}\label{appendix:userstories}

\subsubsection{Role: Admin}

\begin{longtable}{|p{0.1\textwidth}|l|p{0.8\textwidth}|}
\hline \multicolumn{1}{|c|}{\textbf{ID}} &
\multicolumn{1}{c|}{\textbf{Priority}} &
\multicolumn{1}{c|}{\textbf{Story}}  \\
\hline 
\endfirsthead

\multicolumn{3}{c}%
{{\bfseries \tablename\ \thetable -- continued from previous page}} \\
\hline \multicolumn{1}{|c|}{\textbf{ID}} &
\multicolumn{1}{c|}{\textbf{Priority}} &
\multicolumn{1}{c|}{\textbf{Story}} \\
\hline 
\endhead

SA-01 & HIGH & Will be able to create a new contest. When doing so a new web
page should be created, but whether the site should be immediately published or
not is optional. The content of the new site follows a strict template, but
adding a custom css-file will be possible. Each contest has got its own
settings, containing a list of supported compiler profiles, compiler flags,
penalty system, maximum number of contestants, maximum number of contestants
per team, and of course a date and a name. When creating a contest the admin
needs to provide a name and a date, the other settings may be skipped and
default settings will be used.\\ \hline

SA-02 & HIGH & Users are organized in user groups(admin being one of them).
By default three usergroups are provided, admin, judge, contestant and
functionary. The entire solution is based on independent modules of
functionality and each user group has got access to a subset of these modules.
The admin is the only non-modifiable user group, admins have access to all
modules. The admins can modify all other user groups, change permissions of a
group and remove/add member to a group, this includes promoting new admins. The
admins are also able to deactivate users, and even remove them from the
database. \\ \hline

SA-03 & MED & The system is able to gather a large variety of statistics,
what data is to be collected is decided by the admins.\\ \hline


SA-04 & HIGH & The system uses a collection of nodes(computers) for assessing
submissions. The admins can add a node by providing an IP address and the
username and password of a privileged user on that node. These nodes can also
be removed by the admins. The nodes can also be managed in terms of compiler
profile support.\\ \hline

 SA-05 & HIGH & The web page associated with a contest consists of a set of
news items, these can be added by the admin. As with the entire contest web
page the publishing of the news item can be set to a certain date and time.
The news items can also be removed or modified later on.\\ \hline

 
\end{longtable}

\subsubsection{Role: Judge}


\begin{longtable}{|p{0.1\textwidth}|l|p{0.8\textwidth}|}
\hline \multicolumn{1}{|c|}{\textbf{ID}} &
\multicolumn{1}{c|}{\textbf{Priority}} &
\multicolumn{1}{c|}{\textbf{Story}}  \\
\hline 
\endfirsthead

\multicolumn{3}{c}%
{{\bfseries \tablename\ \thetable -- continued from previous page}} \\
\hline \multicolumn{1}{|c|}{\textbf{ID}} &
\multicolumn{1}{c|}{\textbf{Priority}} &
\multicolumn{1}{c|}{\textbf{Story}} \\
\hline 
\endhead

SJ-01 & MED & A judge can submit a problem, where he/she will be able to upload
cases with input/output. He/she can give every case a name. For each problem
the judge can set a resource limit (time + memory) for each compiler profiles.
He/she can upload different solutions that gives the right output, timeout and
the wrong answer. All the solutions should be run-able and produce an output
about the expected result, and if the execution time is inside the given
boundaries. He/she should also be able to check that all problems have
associated solutions that give right and wrong answer, and timeout. \\ 
\hline

SJ-02 & MED & A clarification system will be available to judges, where they
can receive and respond to messages from contestants. When receiving a message,
the judge will get a notification (possible in in the bottom right corner of
the website, [Design choice]). A judge can choose to either send a global
message or a message to a contestant or a team. A global message will be sent
to every contestant in the competition.\\
\hline
\end{longtable}


\subsubsection{Role: Contestant}


\begin{longtable}{|p{0.1\textwidth}|l|p{0.8\textwidth}|}
\hline \multicolumn{1}{|c|}{\textbf{ID}} &
\multicolumn{1}{c|}{\textbf{Priority}} &
\multicolumn{1}{c|}{\textbf{Story}}  \\
\hline 
\endfirsthead

\multicolumn{3}{c}%
{{\bfseries \tablename\ \thetable -- continued from previous page}} \\
\hline \multicolumn{1}{|c|}{\textbf{ID}} &
\multicolumn{1}{c|}{\textbf{Priority}} &
\multicolumn{1}{c|}{\textbf{Story}} \\
\hline 
\endhead
SC-01 & HIGH & A contestant should be registered with an email, name, gender,
and study programme and level. When registered, he/she should receive a
confirmation email. After confirming the account, a contestant should be able
to log in.\\
\hline

 SC-02 & HIGH & When a contestant is logged in he/she will
have access to account information and which teams he/she are invited to, as
well as earlier contests and teams they have participated in. The contestant
should be able to edit account information\\
\hline

 SC-03 & MED & A
clarification system will be available to contestants, where they can ask
questions to the judges. They will also have access to answers the judges have
marked as global.\\
\hline

 \end{longtable}

\subsubsection{Role: Functionary}



\begin{longtable}{|p{0.1\textwidth}|l|p{0.8\textwidth}|}
\hline \multicolumn{1}{|c|}{\textbf{ID}} &
\multicolumn{1}{c|}{\textbf{Priority}} &
\multicolumn{1}{c|}{\textbf{Story}}  \\
\hline 
\endfirsthead

\multicolumn{3}{c}%
{{\bfseries \tablename\ \thetable -- continued from previous page}} \\
\hline \multicolumn{1}{|c|}{\textbf{ID}} &
\multicolumn{1}{c|}{\textbf{Priority}} &
\multicolumn{1}{c|}{\textbf{Story}} \\
\hline 
\endhead


SF-01 & LOW & When a team completes a problem, a table containing the group
name and location should be updated to include this. Each problem has a
corresponding balloon colour. A balloon functionary should be able to register
a balloon colour to each problem.\\
\hline 
\end{longtable}


\subsubsection{Role: Teams}
\begin{longtable}{|p{0.1\textwidth}|l|p{0.8\textwidth}|}
\hline \multicolumn{1}{|c|}{\textbf{ID}} &
\multicolumn{1}{c|}{\textbf{Priority}} &
\multicolumn{1}{c|}{\textbf{Story}}  \\
\hline 
\endfirsthead

\multicolumn{3}{c}%
{{\bfseries \tablename\ \thetable -- continued from previous page}} \\
\hline \multicolumn{1}{|c|}{\textbf{ID}} &
\multicolumn{1}{c|}{\textbf{Priority}} &
\multicolumn{1}{c|}{\textbf{Story}} \\
\hline 
\endhead


ST-01 & HIGH & A contestant must [18.02] be able to register a team, upon
registration he/she is required to input team name, whether or not the team is
onsite, a team password, and a email for the team leader.\\
\hline

 ST-02 & HIGH
& The team leader should be able to edit the team information, invite new
members, and delete the team before the competition. To invite new members you
input their email, and they receive a registration link, where he/she inputs
name, gender and nickname. If the contestant [changed from email 20.02] is
already in the database from a previous competition, the email they receive
contains a confirmation link. Every contestant can manage the team they are a
member of. All informations is editable in the team overview which can be
reached from a contestants login. A confirmation email is sent to the edited
user.\\
\hline

 ST-03 & MED & A team should be able to deliver submissions to
problems, and get a response from the system. The response should be whether
the submission is right, wrong, or gives timeout.\\
\hline


\end{longtable}

