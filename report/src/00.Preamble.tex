\section*{Foreword}
Originally inspired by the Nordic Collegiate Programming Contest (NCPC),
it has been held at NTNU every spring since 2007. The format is a
five-hour contest with competing teams consisting of one, two or three
contestants. A team of volunteer judges write the problems and answer
clarification requests during the contest, while another team hands out
balloons for each solved problem. Usually a rather hectic affair, it is
extremely important that everything is well prepared. The number of
teams is often more than 100, with the record being 162 teams in 2011

The contest system that verifies solutions is at the heart of the
contest when it is in progress, and needs to be working perfectly at
all times. The system must handle several submissions per second, while
verifying that each one is correct and runs within the set resource
limits. Submissions must show up on the high score list, and when
problems are solved the team handing out balloons must be notified. In
addition to this there were a lot of other functional requirements
having to do with the bureaucracy of organizing the contest

A requirement was that new features could be easily added in the future,
and the code was written with this in mind. The project will now become
open source, and all programming contest enthusiasts will soon be able
to request and implement their desired features

All aspects of this project have been pleasing and delightful for us.
The team has exceeded all our expectations and their system will be
used for years to come.

\hfill -- \textit{Christian Chavez, IDI Open Manager}

\pagebreak

\section*{Preface}

Before there were computers, there were algorithms. But now that there are
computers, there are even more algorithms, and algorithms lie at the heart of
computing. Designing a system for eager students to hone their skill in the
heart of computing has been a true joy.

Our group never wanted to settle for adequacy and mere requisiteness. For the
past few months, weve taught ourselves a new programming language and framework
and used advanced development frameworks - while tackling many social and
technical conflicts.

We have proven how ``Ambition is a dream with a V8 engine'', as Elvis Presley
once said.

The group would like to thank our eager customers, Finn Inderhaug Holme,
Christian Chavez and Christian Neverdal Jonassen for their time to meet us and
provide constructive feedback. We also owe a big thanks to our supervisor, Hong
Guo, for constructive criticism and reflections; without which, we would not
ascertain the peak of our own potential.
