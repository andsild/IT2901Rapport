\chapter{Risk List}\label{appendix:risk_list}

This appendix includes tables including all the risks we considered for this project.
To structurize our risk register, we divided each into the following
categories:
\begin{itemize}
    \item \textbf{Budget risks} are all risks that can be associated with
        financial aspects of our project.
    \item \textbf{Organizational risks} are those that might arise because of
    group structure and task delegation.
    \item \textbf{People Management} comprises all risks associated with team
        management and each individual in the group.
    \item \textbf{Requirements risks} are related to errors in requirements
        engineering.
    \item \textbf{Schedule risks} are about meeting deadlines and task
        delegation.
    \item \textbf{Technology and tools}; product talk about technical risks that
        might arise with tools and our product.
\end{itemize}

To prioritize our risks, we have also given each risk a probability,
consequence and total risk, abbreviated Pr, C, TR, respectively. Each
of these were assigned values from 1-10, where 10 indicated
``very high''. A 10 translates to
the following for each field:
\begin{itemize}
    \item \textbf{Consequence}: event of risk will be fatal to our project.
    \item \textbf{Probability}: risk will probably happen
    \item \textbf{Total risk}: The risk is a big threat and should be monitored closely.
\end{itemize}

Total risk is calculated as Consequence x Probability. By multiplying
these numbers, we get a sorted list of the most dangerous risks. 


\newgeometry{left=2cm}
\begin{landscape}

\section{People Management}
\begin{tabular}{|>{\columncolor{CadetBlue}}p{3.5cm}|>{\columncolor{CadetBlue}}p{1.1cm}
        |>{\columncolor{Mahogany}}p{.3cm}|>{\columncolor{Mahogany}}p{.3cm}|>{\columncolor{Mahogany}}p{.3cm}
        |>{\columncolor{Orange}}p{5.2cm}|>{\columncolor{Orange}}p{6.2cm}|}% 7
\hline
\rowcolor{White}\textbf{Description}&\textbf{\#ID}
        &\textbf{Pr}&\textbf{C}&\textbf{TR}
        &\textbf{Preventative action}&\textbf{Remedial action}\\
\hline
Personal argument&PM-01&8&5&40&Frequent meetings and social events&Open discussion \\
\hline
Dependency on team member&PM-02&6&6&36&Short sprints and team members usually work in groups of two&New meeting where we consider a redistribution of WP \\
\hline
Underburdened team-member; slack&PM-03&7&4&28&Keeping track of the work done by
each member as well as the number of hours spent on any given WP.\ In the
beginning of the sprint focus more on an evenly distributed workload among
team members.&If the team-member continues to slack put it on the agenda for the next meeting and allow the team-member to explain his/her reasons for slacking. \\
\hline
Team members are late&PM-04&9&2&18&If you are late, you need to bring a cake or cookies to the next meeting&You need to bring a cake or cookies, and if it happends several times, an extraordinary meeting will be called, where new consequences will be discussed. \\
\hline
Team member is not qualified for any assigment&PM-05&4&7&28&Try to keep every member up to date on the entire system by not letting anyone work for too long on the same part of the system.&Add unqualified member to an existing pair working on a WP. \\
\hline
Miscommunication&PM-06&7&3&21&Frequent meetings with discussion about team letting all team members try different areas in the application&As per SDLC; evaluation, analysis, re-start assigment \\
\hline
Dependency on external person&PM-07&3&6&18&Frequent communication with the customer. &Well-planned sprints with a low level of dependency between WPs. \\
\hline
Displacement; team members do not feel comfortable in group&PM-08&2&7&14&Social events. &Talk to our supervisor and ask for suggestions \\
\hline
Overburdened team-member&PM-09&4&2&8&Short sprints and small WPs. A team member will only be assigned to a few WPs at a time.&Frequent meetings where WPs can possibly be redistributed. \\
\hline
\end{tabular}

\section{Budget}
\begin{tabular}{|>{\columncolor{CadetBlue}}p{3.5cm}|>{\columncolor{CadetBlue}}p{1.1cm}
        |>{\columncolor{Mahogany}}p{.3cm}|>{\columncolor{Mahogany}}p{.3cm}|>{\columncolor{Mahogany}}p{.3cm}
        |>{\columncolor{Orange}}p{5.2cm}|>{\columncolor{Orange}}p{6.2cm}|}% 7
\hline
\rowcolor{White}\textbf{Description}&\textbf{\#ID}
        &\textbf{Pr}&\textbf{C}&\textbf{TR}
        &\textbf{Preventative action}&\textbf{Remedial action}\\
\hline
Maintenance costs exceed expectations&B-01&5&3&15&Use highly maintainable frameworks as much as possible, and stick to Open Source as much as possible.&Optimizing code base in hopes of increasing maintainability.\\
\hline
Third party plugin demands more money than initially expected&B-02&2&3&6&We've got a green light for putting GentleIDI under the GNU Public License, which means that we have got free access to software under GPL.&Look for alternative plugins.\\
\hline
Unexpected need for non-free third-party service&B-03&3&3&9&Extensive research on tools needed,before we decide on what we are going to use.&Look for alternative free third-party services\\
\hline
Maintenance requires access to tools/environments that cost money&B-04&2&3&6&Use highly maintainable frameworks as much as possible,and stick to Open Source as much as possible.&Request customer meeting to solve the issue.\\
\hline
\end{tabular}
\section{Schedule}

\begin{tabular}{|>{\columncolor{CadetBlue}}p{3.5cm}|>{\columncolor{CadetBlue}}p{1.1cm}
        |>{\columncolor{Mahogany}}p{.3cm}|>{\columncolor{Mahogany}}p{.3cm}|>{\columncolor{Mahogany}}p{.3cm}
        |>{\columncolor{Orange}}p{5.2cm}|>{\columncolor{Orange}}p{6.2cm}|}% 7
\hline
\rowcolor{White}\textbf{Description}&\textbf{\#ID}
        &\textbf{Pr}&\textbf{C}&\textbf{TR}
        &\textbf{Preventative action}&\textbf{Remedial action}\\
\hline
    Prestudies require more time than anticipated&S-01&9&7&63&We have a WP for pre-studies, and have included it in our sprints&Revise our WBS, and possible have an increased workload/work-hours in the following sprints, so we don't fall behind our schedule. \\
    \hline
    Failure to meet requirements on time&S-02&5&8&40&WBS, milestones plan and short sprints (1 or 2 weeks) allow us to focus on deadlines, and continously see our work progress&Have extraordinary meetings with supervisor and the customer to discuss the further development of the project. Be apologetic towards the customer, and come up with a new plan, that the customer is satisfied with.\\
    \hline
    Sprint-estimations are off&S-03&9&5&45&The whole group participate in planning a sprint, and estimating each task&Re-adjust our estimations in the next sprint, and in that way learn from our mistakes.\\
    \hline
    Failure to deliver sufficient documentation on time&S-04&5&6&30&WBS, milestones plan and short sprints (1 or 2 weeks) allow us to focus on deadlines, and continously see our work progress&Meetings with supervisor and customer, agree upon a new deadline, and increase the workload the following days to we meet the deadline.\\
    \hline
    Need for extra technology / features that requires training to use&S-05&3&6&18&We use extensive frameworks who has a lot of documentation, which makes it easier to learn.&Adjust the WBS and our sprints so we take into account that we need more time to learn new technology. Focus on this in the coming sprint planning.\\
    \hline
\end{tabular}

\section{Organizational}
\begin{tabular}{|>{\columncolor{CadetBlue}}p{3.5cm}|>{\columncolor{CadetBlue}}p{1.1cm}
        |>{\columncolor{Mahogany}}p{.3cm}|>{\columncolor{Mahogany}}p{.3cm}|>{\columncolor{Mahogany}}p{.3cm}
        |>{\columncolor{Orange}}p{5.2cm}|>{\columncolor{Orange}}p{6.2cm}|}% 7
\hline
\rowcolor{White}\textbf{Description}&\textbf{\#ID}
        &\textbf{Pr}&\textbf{C}&\textbf{TR}
        &\textbf{Preventative action}&\textbf{Remedial action}\\
\hline
    No person has responsibility for an assigment, although it is believed to be delegated&O-01&8&6&48&Strict use of the activity plan. The activity plan should be kept consistent at all times, this way all members know what the others are doing at any given time.&When discovered the given WP should be marked as unallocated in the activity plan and treated like any other WP in the sprint.\\
    \hline
    Project is, at current point not satisfactory, and it is hard to understand why&O-02&6&7&42&Writing meeting summaries, and in general keeping track of what is being done and how.&Review what work has been done up untill that point, how it has been done, and try to find a solution to the problem.\\
    \hline
    Bottleneck; in order for team-members to advance, other team members must finish their work&O-03&7&7&49&Try to avoid dependencies between WPs when setting up sprints. In case of such dependencies being unavoidable these WPs should be scheduled at the beginning of the sprint.&Delegate or even create new WPs to the team members currently being idle.\\
    \hline
    A task is delegated to more than one person&O-04&2&3&6&Strict use of the activity plan. The activity plan should be kept consistent at all times, this way all members know what the others are doing at any given time.&The two members should discuss how the issue should be solved, and update the activity plan according to that.\\
    \hline
\end{tabular}

\section{Tools and tools; product}
\begin{tabular}{|>{\columncolor{CadetBlue}}p{3.5cm}|>{\columncolor{CadetBlue}}p{1.1cm}
        |>{\columncolor{Mahogany}}p{.3cm}|>{\columncolor{Mahogany}}p{.3cm}|>{\columncolor{Mahogany}}p{.3cm}
        |>{\columncolor{Orange}}p{5.2cm}|>{\columncolor{Orange}}p{6.2cm}|}% 7
\hline
\rowcolor{White}\textbf{Description}&\textbf{\#ID}
        &\textbf{Pr}&\textbf{C}&\textbf{TR}
        &\textbf{Preventative action}&\textbf{Remedial action}\\
\hline
    End product is not satisfactory&TT-01&2&9&18&Customer meetings regularly,
    and keeping in contact through e-mail aswell. Give the customer access to
    our git-repository, so they have access to our source code, and also
    perform different type of tests (user-testing, etc) &
    Call in to a meeting with our supervisor, and our customer. Explain what went wrong, apologize and deliver our documentation.\\
    \hline
    Tools used for development are not suitable / efficient in later parts of the project&TT-01&2&8&16&Researching the tools we use, and planning ahead. Development planning allow us to discover problems before they appear.&Look for alternative tools. If changing tools involve a lot of work, and changes to the project, decide in a meeting if we want to continue with the inefficient tools, or if we want to make the change.\\
    \hline
    Problems with integrating components&TT-03&7&3&21&Have extensive system documentation and planning. Involve the whole group in the process.&Re-evaluate our system architecture, and look for solutions that won't affect other parts of the system.\\
    \hline
    Other solutions available make our product less desirable&TT-04&1&8&8&Do thorough work on the system requirements in hopes of providing a system well-tailored to the customer's needs.&Reevaluate the requirements.\\
    \hline
    Network cannot deal with traffic&TT-05&1&8&8&Keep optimization in mind when developing.&Try to find redundant data being sent possibly apply use of compression.\\
    \hline
    Submitted program has access to resources&TT-06&5&5&25&Submitted programs are to be run by a sandbox-user with a very restricted set of resources available.&Review code in hopes of finding the bug. \\
    \hline
    Platform / hardware unavailible, such that testing is difficult&TT-07&2&5&10&We use services provided by companies known to provide good system uptime. Most of our tools are hosted by Red Hat.&Setup temporary development environment.\\
    \hline
    Tools used in initial development are not available after release, and future developers have difficulty extending product&TT-08&2&3&6&Make sure requirements are written properly, understood properly, succint, etc&Document our work, so it is easy for future developers to understand the system. \\
    \hline
    Database cannot handle amount of transactions&TT-09&1&4&4&Keep optimization in mind when developing.&Optimize code in order to lower amount of transactions.\\
    \hline
    A tool does not perform the functions it was intended for&TT-010&2&3&6&Learn the tools properly, and read the documentation provided with each tool.&Look for alternative tools.\\
    \hline
\end{tabular}

\section{Requirements}
\begin{tabular}{|>{\columncolor{CadetBlue}}p{3.5cm}|>{\columncolor{CadetBlue}}p{1.1cm}
        |>{\columncolor{Mahogany}}p{.3cm}|>{\columncolor{Mahogany}}p{.3cm}|>{\columncolor{Mahogany}}p{.3cm}
        |>{\columncolor{Orange}}p{5.2cm}|>{\columncolor{Orange}}p{6.2cm}|}% 7
\hline
\rowcolor{White}\textbf{Description}&\textbf{\#ID}
        &\textbf{Pr}&\textbf{C}&\textbf{TR}
        &\textbf{Preventative action}&\textbf{Remedial action}\\
\hline
    Major change to requirements&R-01&5&4&20&Customer meetings regularly where we agree upon a requirement specification.&New customer meeting where we re-evaluate the requirements specification,and which priorities each requirement has.\\
    \hline
    Customer fails to understand impact of requirements&R-02&2&7&14&Customer meetings regularly where we agree upon a requirement specification.&Customer meeting where we explain the impact of the requirement,and get the customer to explain their requirements that we have different opinions on.\\
    \hline
    Finished product does not meet requirement&R-03&1&9&9&Customer meetings,they have access to our git-repository where our source code is & Test-events where they can test the functionality. Finish our documentation,and pass it on to other developers. Apologize to the customer.\\
    \hline
    Failed interpretation of requirement&R-04&3&4&12&Customer meetings regularly where we agree upon a requirement specification.&Customer meeting where we re-discuss the requirement specification,and make sure we understand what the customer wants.\\
    \hline
\end{tabular}
\end{landscape}
\restoregeometry
