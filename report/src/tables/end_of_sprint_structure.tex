\chapter{End of Sprint Structure}\label{appendix:endOfSprint}

\begin{framed}
\textbf{Meeting Agenda: }
\begin{itemize}
    \item Daily Scrum
    \begin{itemize}
        \item What have you done since last time?
        \item Have you had any obstacles? 
    \end{itemize}

    \item Three good/bad things
    \begin{itemize}
        \item All team members take turns saying three good and three negative things
        about the previus sprint. 
        \item This is done without interruptions
        \item If someone brought a cake, serve it here.
    \end{itemize}

    \item Show what has been done
    \begin{itemize}
        \item Every group member take turns showing what they have completed. 
        \item Discuss what has not been done
    \end{itemize}
        
    \item Sprint end meetings
    \begin{itemize}
        \item Effecitvely disucss what could have been done better
    \end{itemize}

    \item Other
    \begin{itemize}
        \item If someone want to talk about something this is the time.
    \end{itemize}

    \item Sprint planning meeting
    \begin{itemize}
        \item Select work that has to be done
        \begin{itemize}
            \item The work is selected from the release backlog and put into to
                sprint backlog
        \end{itemize}
        \item Break these into smaller task/activities
        \item Give each of these tastk/activities a priority 
        \item Give each of these task/activities a time approximation
        \item Distribute on task/activite to each member. 
    \end{itemize}
\end{itemize}

\textbf{About time estimation}
\begin{itemize}
    \item When voting for how long time a task/acivity will take, only powers of two are allowed:
        \begin{itemize}
            \item 2, 4, 8, 16, 32, 64 etc.
            \item 8 is characterized as a day
        \end{itemize}
\end{itemize}

\textbf{About prioritzing the task/activites}
\begin{itemize}
    \item Options when voting are 1, 2, 3 where 1 means LOW, 2 mean MEDIUM and
        3 means HIGH. 
\end{itemize}

\textbf{General}
\begin{itemize}
    \item All members has a vote. 
    \item If one estimates/prioritize different than the other members, he can,
        if he want to, tell the group why he estimated as
        he did. A new estimation will then take place.  
\end{itemize}
\end{framed}
