\chapter{Risk Management Framework}
A risk is an event or condition that, if it occurs, could have a negative
effect on a project's objectives. To avoid these risks, and to be able to deal
with them effectively, we established a risk modelling framework Our
framework is based upon our own experience and examples from the many documents
that exists on the subject.

By explicitly writing down corresponding actions for risks that occur, we could
deal with risks without disagreements. It also let external parties get an
overview of what risks we are aware of, and how we reviewed them. The external
party can then notify us of unknown risks or modifications to our priorities. 

\section{Terminology and Categories}

To structurize our risk register, we divided each into the following
categories:
\begin{itemize}
    \item \textbf{Budget risks} are all risks that can be associated with
        financial aspects of our project.
    \item \textbf{Organizational risks} are those that might arise because of
    group structure and task delegation.
    \item \textbf{People Management} comprises all risks associated with team
        management and each individual in the group.
    \item \textbf{Requirements risks} are related to errors in requirements
        engineering.
    \item \textbf{Schedule risks} are about meeting deadlines and task
        delegation.
    \item \textbf{Technology and tools}; product talk about technical risks that
        might arise with tools and our product.
\end{itemize}

To prioritize our risks, we have also given each risk a probability,
consequence and total risk, abbreviated Pr, C, TR, respectively. Each
of these were assigned values from 1-10, where 10 indicated
``very high''. A 10 translates to
the following for each field:
\begin{itemize}
    \item \textbf{Consequence}: event of risk will be fatal to our project.
    \item \textbf{Probability}: risk will probably happen
    \item \textbf{Total risk}: The risk is a big threat and should be monitored closely.
\end{itemize}

Total risk is calculated as Consequence x Probability. By multiplying
these numbers, we get a sorted list of the most dangerous risks. 

\section{Scope of Risk Assessment}
Finding the right balance to the extent of documentation is difficult.
Extensive risk-frameworks can consume more hours in maintenance than
they save. To deal with our lacking experience, we only wanted to
document the most likely risks. To us, this meant only including risks
with a total risk value of more than 30

We considered specifying additional information to each risk, like
context and associated risks. However, we felt every member of the
group had a similar understanding of the risks, so writing this
information down would be superfluous. In addition, since the risks
were orally reviewed, we did not want to rely too much on what had been
written down.

\section{Risk Identification}
We tried to involve every group member in the making of the risk
register. The estimates from 1 to 10 were assigned based on our own
experience from previous projects. The list was filled out by three
members of the group, and then later presented to the whole group for
reviewal and agreement on the values. 

Risks that became known in later parts of our development was promptly
added to our risk register. We expected few of these, and few did
occur, so we have not performed any revision control. Our means of identifying
risks was through discussions and agreements that we were not
performing optimally.

\section{Risk Monitoring}
Our primary method for surveilling risks was weekly discussions. In
these meetings, we had open discussions of the group's
progress and development. In addition, we had one monthly meeting where
we would discuss the risks more thorough and in-depth. This involved
re-discussion of the group's expectations and our
involvement in the project. These monthly meetings were referred to as
``snapshots''. The snapshots
specifically addressed the problem that many projects start out quite
ambitiously, but tend to deteriorate, something we wanted to avoid.

To avoid groupthink\footnote{The concept of trying to avoid conflict
by not speaking one's mind. For more, see:
http://www.psysr.org/about/pubs\_resources/groupthink\%20overview.htm}
and complacency, we required each group member on our weekly meetings
to mention three good and three negative points. After that, each
member could bring up extra topics for discussion. For each discussion,
we made sure to be conclusive by explicitly writing how to deal with a
given problem. 

We have frequently involved the supervisor and customer in our process.
We made sure to ask for insights on our development progress. After
each meeting we also wrote down meeting minutes and a summary. This was
later sent to the respective party to ensure agreement on what had been
concluded in the meeting.

\section{Complete List of Risks}
We have chosen to put the complete list in appendix~\ref{appendix:risk_list}.
