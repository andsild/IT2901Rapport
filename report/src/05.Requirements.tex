\section{Requirements Specification}
According to the gantt chart (Fig 4.1) the team were supposed to update the
requirement specification starting from week 2 and continuing up until week
10. For us it was still the case that there were a clearly identifiable
requirement specifications phase. This was primarily from week 2 up to and
including week 4. The outcome from this three week process was heavily used in
order to establish agreement between us and the customer. This chapter
presents the result from this process. 

\subsection{Purpose and Scope of this Specification}
The purpose of the requirement specification document is to specify the
objectives for our end product. Requirements are written at different
levels of detail. This is to make it easy to communicate the requirements to
both business and technical parties. We have mainly written the functional
requirements as stories and then broken them into smaller pieces. This makes
the requirements easy to communicate to the customer, and succinct for the
developers. These stories can be viewed in appendix TODO.\ It is important to
recognize that our project only lasted for a few months. Thus, late changes
to requirements were inserted promptly and without revision control. This is
a common practice in agile development\footnote{\ Page 91, Sommersville}.
The advantage and reason we chose not to perform revision control, is that we
could save time in not formally documenting all changes.

The coverage of the requirements is intended to be a complete coverage
of the product. This implies that all features available from the
application domain is listed in our specification. What the requirements
specification does not cover are organizational and external requirements. This
follows from the small amount of administrative users and developers involved,
and trust between the customer and the developers.

\subsection{Process of the Requirement Specification}
The customer passed on an initial list of requirements to our group. After a
classification and organization of the features, we drafted scenarios and
internally discussed the implication to each requested feature. Therein, we saw
what features would be infeasible and additional features we would want to
introduce to the customer. The modified list of requirements was then presented
to the customer, before proceeding with the implementation of the end-product.
Throughout the entire development process both we and the customer have been
modifying the list of requirements.

\subsection{Product/service description}
In this section, you will find our interpretation of the physical user-domain.
The reader should note that some members of our group has competed earlier,
which has given us helpful empirical insight.

\subsubsection{Expected Physical Environment}
Our solution is used in different contexts. Table X.X has the different
application and user-domains.

\begin{tabular}{|m{3.1712599in}|m{3.1712599in}|}
\hline
IDI Open is hosted in P15, Høgskoleringen 3, on
Gløshaugen campus every year. Every team participating in the
contest get allocated their own computer. &
For offsite contestants, javascript must be enabled.\\
\hline
Software is required. A web server(Apache, Nginx), database
server(MySQL, PostgreSQL), Python with PyPi package manager.
 &
Linux kernel with ssh enabled, supplemented with a root user.
\\\hline
\end{tabular}

\subsubsection{User Characteristics}
Table X.X show different stereotypes of expected typical users. While open to
deviations from the stereotypes, they highlight important properties required
for our solution.

\begin{tabular}{|m{2.9837599in}|m{3.1087599in}|}
\hline
\begin{itemize}
    \item Irresponsive interfaces
    \item Incorrect data
\end{itemize}
\begin{itemize}
    \item User submission system
    \item Response types
\end{itemize}
 &
\begin{itemize}
    \item Irresponsive interfaces
    \item Node failures
    \item Incorrect data
\end{itemize}

\begin{itemize}
\item Backend system\item 
Dataflow
\end{itemize}
\\\hline
keep track of score
\begin{itemize}
\item Irresponsive interfaces
\item
Lack of overview
\end{itemize}
\begin{itemize}
\item Backend system
\item 
Dataflow
\end{itemize}
 &
\begin{itemize}
\item Dissatisfied contestants\item
No overview\end{itemize}
\begin{itemize}
\item Nothing special\end{itemize}
\\\hline
and information
\begin{itemize}
\item Mis-information
\end{itemize}
\begin{itemize}
\item Scoreboards, about competition
\end{itemize}
 &
\\
\hline
\end{tabular}

It can be seen in table X.X that the most prominent trait of our users is that
they have a background in computer science. As a consequence, it is assumed a
higher level of technical competence from our users. The user profiles also
highlight that some features were more important than others, e.g.\
responsiveness over aesthetics.

\subsection{Requirements}
Stories can be ambiguous and open for misinterpretation.
we felt that a natural language specification of requirements would make it
easier to understand our application domain. To reduce miscommunication we made
sure to give each specification as short, succinct sentences. The stories were
used as a way to communicate with the customer about requirements without them
having to read through the table of requirements.

There are three different states for priorities, HIGH, MED and LOW.\
This ensured strict priorities.
Using more states would make it hard to differentiate between the priorities we
gave the requirements.

The following definitions make out the guideline for
prioritizing the requirements:
\begin{itemize}
    \item HIGH: The requirement is a ``must have''. To have a successfull product,
        the requirement must be implemented.
    \item MED: The requirement is a ``should have''. The fulfillment of the
        requirement will benefit the quality system.
    \item LOW: The requirement is a ``nice to have''. This includes functionality
        not critical to the system.
\end{itemize}

\subsubsection{Functional}
The functional requirements are broken down in different categories.
Each category corresponds to a user group. The categories are Admin, Judge,
Contestant, Functionary, Teams, and Other. Each category has an ID, priority
and story. Table X.X shows the complete list of the requirements, while the
corresponding stories are given in appendix\ref{appendix:userstories}

The ID system can be interpreted in the following way
\begin{itemize}
    \item The F stands for Functional
    \item The second letter determines which category, e.g A stands for admin.
\end{itemize}

The milestone show when each requirement needs to be met.

\subsubsection{Functional requirements for Admin}
\begin{supertabular}{|p{2cm}|l|m{0.33795986in}|m{1.2684599in}|m{0.5254598in}|m{1.0983598in}|m{1.0566599in}|}
\hline
Requirement & ID & Story & Comments & Priority & Milestone & Test \\
\hline
An admin shall be able to create a new contest & FA-01 & SA-1 & A new contest
equals a new web page & HIGH & M-03 &
TF-01\\
\hline

An admin can choose whether the site should be published immediately or not &
FA-02 & SA-1 & & MED & M-03 & TF-01\\
\hline

An admin can add custom CSS to the web-page & FA-03 & SA-1 & & LOW & M-03 & TF-01\\
\hline

An admin shall be able to choose settings for the contest & FA-04 & SA-1 & of
contestants, maximum number of contestants per team, date, name.  Default
settings will be provided & HIGH & M-06 & TF-03\\
\hline

An admin shall have access to all modules in the program & FA-05 & SA-2 & &
HIGH & M-06 & TF-03\\ 
\hline

An admin can change permission of a usergroup & FA-06 & SA-2 & & LOW & M-06 &
TF-04\\ 
\hline

An admin can remove/add to a user group. & FA-07 & SA-2 & This
includes promoting new admins & LOW & M-06 & TF-04\\ 
\hline 

An admin can deactivate users & FA-08 & SA-2 & & LOW & M-06 & TF-04\\ 
\hline

An admin can remove users from the database & FA-09 & SA-2 && HIGH & M-06 & TF-04\\ 
\hline

An admin can add a node & FA-10 & SA-4 & The node must be a privileged
user & HIGH & M-06 & TF-06\\ 
\hline 

An admin can remove a node & FA-11 & SA-4 & & HIGH & M-06 & TF-06\\ 
\hline 

An admin can manage a node. & FA-12 & SA-4 & This requirement is in terms of
compiler profiles support & HIGH & M-06 & TF-06\\ 
\hline 

An admin can add more than one node & FA-13 & SA-4 & & MED & M-06 & TF-06\\ 
\hline 

An admin can add news items & FA-14 & SA-5 & & HIGH & M-03 & TF-01\\ 
\hline

An admin can remove new items & FA-15 & SA-5 & & MED & M-03 & TF-01\\ 
\hline 

An admin can modify news item & FA-16 & SA-5 & & MED & M-03 & TF-01\\ 
\hline 
\end{supertabular}

\subsubsection{Functional requirements for Judge}
\begin{supertabular}{|p{2cm}|l|m{0.33795986in}|m{1.2684599in}|m{0.5254598in}|m{1.0983598in}|m{1.0566599in}|}
\hline
A Judge can create a problem & FJ-01 & SJ-1 & This includes cases with input
and output & HIGH & M-06 &
TF-05\\\hline

A judge can upload cases to a problem and name each case & FJ-02 & SJ-1 &
 & MED & M-06 & TF-05\\\hline

A judge can set a resource limit on each task & FJ-03 & SJ-1 & & LOW & M-06 &
TF-05\\
\hline

A judge can add a solution that gives the right output & FJ-04 & SJ-1 & & HIGH
& M-06 & TF-05\\
\hline

 A judge can add a solution that gives timeout & FJ-05 &
SJ-1 & & MED & M-06 & TF-05\\
\hline

 A judge can add a solution that gives wrong
answer & FJ-06 & SJ-1 & & MED & M-06 & TF-05\\
\hline

 A judge shall be able to
view and edit all problems & FJ-07 & SJ-1 & & HIGH & & TF-05\\
\hline

 A judge
shall be able to respond to a question from a team & FJ-08 & SJ-2 & This is
about the clarification system. & MED & M-06 & TF-07\\
\hline

 A judge shall get
a notification when received a question & FJ-09 & SJ-2 & & LOW & M-06 &
TF-07\\
\hline

 A judge shall be able to respond to a question globally & FJ-10 &
SJ-2 & By globally it is intended that the all teams can view the response and
question & HIGH & M-06 & TF-07\\
\hline

 A judge shall be able supervise all
submissions & FJ-11 & & & & M-06 & \\
\hline

 \end{supertabular}

\subsubsection{Functional requirements for Contestant}

\begin{supertabular}{|m{1.1191599in}|m{0.35875985in}|m{0.33795986in}|m{1.2580599in}|m{0.5045598in}|m{1.2476599in}|m{1.0670599in}|}
\hline
A contestant shall be able to edit their own information & FC-01 & SC-1 & &
HIGH & M-03 & TF-08\\\hline When created a contestant shall receive a
confirmation email & FC-02 & SC-1 & & HIGH & M-03 & TF-08\\\hline A contestant
shall see which teams they are invited to & FC-03 & SC-2 & & HIGH & M-03 &
TF-09\\\hline A contestant shall see which team they are a member of & FC-04 &
SC-2 & & HIGH & M-03 & TF-02\\\hline A contestant shall see which teams and
contests they have participated in earlier & FC-05 & SC-2 & & MED & M-03 &
TF-09\\ \hline A contestant shall be able to ask a question to a judge & FC-06
& SC-3 &
 & MED & M-03 & TF-07\\\hline A contestant shall have access to global answers
 from judges & FC-07 & SC-3 & & MED & M-06 & TF-07\\\hline A contestant shall
 be able to change his/her email & FC-02 & SC-2 & & MED & & \\\hline
\end{supertabular}

\subsubsection{Functional requirements for Functionary}

\begin{tabular}{|m{1.1191599in}|m{0.33795986in}|m{0.33795986in}|m{1.2580599in}|m{0.5045598in}|m{1.1191599in}|m{1.0775598in}|}
\hline A functionary shall be able to register a balloon colour to each
task/problem & FF-01 & SF-1 &
 & LOW & M-06 & TF-12\\\hline A functionary shall have access to information
about newly completed problems & FF-02 & SF-1 &
 & MED & M-06 & TF-12\\\hline
\end{tabular}

\subsubsection{Functional requirements for Teams}
\begin{supertabular}{|m{1.1191599in}|m{0.34835985in}|m{0.33795986in}|m{1.2684599in}|m{0.5045598in}|m{1.1087599in}|m{1.0670599in}|}
\hline
A user shall be able to register a team & FT-01 & ST-1 & Whether or not the
team is onsite, a team password, and a email for the team leader & HIGH & M-06
& TF-02\\\hline A user shall be able to register other team members for the
team & FT-02 & ST-2 & By providing other users' email & HIGH & M-03 &
TF-02\\\hline If the contestant is already in the system shall recognize
personal info & FT-03 & ST-2 & Personal information like name, gender and so
on.  & LOW & M-03 & TF-10\\\hline A team leader must be able to invite new
members & FT-04 & ST-2 & Input: email & MED & M-03 & TF-02\\\hline A team
leader should be able to delete the team before the competition & FT-05 & ST-2
& & MED & M-03 & TF-10\\\hline When a team leader invites a new member the new
member must receive a registration link & FT-06 & ST-2 & The receiver of this
email link must fill in the data specified in: T-3 & MED & M-03 & TF-02\\\hline
If a member's email is already in the database they will receive a confirmation
link & FT-07 & ST-2 & The confirmation link will include automatically filled
data. See T-4 & LOW & M-03 & TF-10\\\hline All team information is editable in
the team overview. & FT-08 & ST-2 &
 & LOW & M-03 & TF-10\\\hline A team must be able to deliver submissions to
problems & FT-09 & ST-3 &
 & HIGH & M-06 & TF-11\\\hline When a team deliver a submission they shall
receive response from the system & FT-10 & ST-3 & system should give timeout.
This is specified by a judge.  & HIGH & M-06 & TF-11\\\hline
\end{supertabular}

\subsubsection{Other requirements}
\begin{supertabular}{|m{1.1191599in}|m{0.35875985in}|m{0.33795986in}|m{1.2580599in}|m{0.5045598in}|m{1.1087599in}|m{1.0670599in}|}
\hline
The system shall be able to gather some statistics & FO-01 & SA-3 & It is here
implied statistics from contestants in accordance with FE-3 & HIGH & M-05 &
\\\hline The system shall be able to gather a large variety of statistics
specified by the admin & FO-02 & SA-3 &
 & LOW & M-05 &
\\\hline The system shall include a clarification system & FO-03 & SJ-2 & This
is according to FJ-8, FJ-9, FJ-10, and FE-14, FE-15, FE-16, FE-17, FE-18 & HIGH
& M-07 & TF-07\\\hline The contest results are to be visible in the form of a
highscore list. & FO-04 & ST-03 &
 & MED & M-07 &
\\\hline
\end{supertabular}

\subsection{Non-functional}
The nonfunctional requirements defines, what objectives our end product
needs to meet. The qualitative measures make it easier to agree on
whether the requirement is fulfilled or not. Table can be interpreted in the following way:
\begin{itemize}
    \item In ID the NF stands for Non-Functional
    \item Measure describes how/what the property should measure.
\end{itemize}

\subsubsection{Speed}
\begin{supertabular}{|l|p{5cm}|p{1.8cm}|l|p{4.4cm}|}
\hline
\textbf{ID:} & \textbf{Measure:} & \textbf{Value:} & \textbf{Priority:} & \textbf{Comment:}\\
\hline

NF-01 & Response from action & {\textless} 1.5 sec & MED & \\
\hline

NF-02 & Posting news & {\textless} 5 sec & MED & \\
\hline

NF-03 & Edit user & {\textless} 1 min & MED & Time from submission until the
whole system is updated\\
\hline
\end{supertabular}

\subsubsection{Size}
\begin{supertabular}{|l|p{5cm}|p{1.8cm}|l|p{4.4cm}|}
\hline
\textbf{ID:} & \textbf{Measure:} & \textbf{Value:} & \textbf{Priority:} & \textbf{Comment:}\\
\hline
NF-04 &
Number of contestants &
500 &
HIGH &
\\\hline

NF-05 &
Number of teams &
200 &
HIGH &

\\\hline

NF-06 &
Number of judges &
20 &
HIGH &

\\\hline

NF-07 &
Number of admins &
{\textgreater} 1 &
HIGH &

\\\hline

NF-08 &
Limitation of solution size &
50kB &

\\\hline
\end{supertabular}

\subsubsection{Ease of Use}

\begin{supertabular}{|l|p{5cm}|p{1.8cm}|l|p{4.4cm}|}
\hline
\textbf{ID:} & \textbf{Measure:} & \textbf{Value:} & \textbf{Priority:} & \textbf{Comment:}\\
\hline
NF-09 & Learning time for contestants & {\textless} 5 min & MED & The users of
the program should be good at computers and therefore know what they are
doing.\\
\hline

NF-10 & Learning time for admins & {\textless} 15 min & MED &
\\\hline

NF-11 &
Learning time for judge &
{\textless} 10 min &
MED &
\\\hline

\end{supertabular}
\subsubsection{Reliability}
\begin{supertabular}{|l|p{5cm}|p{1.8cm}|l|p{4.4cm}|}
\hline
\textbf{ID:} & \textbf{Measure:} & \textbf{Value:} & \textbf{Priority:} & \textbf{Comment:}\\
\hline

NF-12 & Mean time to failure &
{\textgreater} 1 week & HIGH & The program should NOT be down during a
contest\\
\hline

NF-13 &
Availability &
{\textgreater} 99.9\% &
HIGH &
How much is the software up and running.\\
\hline


\end{supertabular}
\subsubsection{Robustness}
\begin{supertabular}{|l|p{5cm}|p{1.8cm}|l|p{4.4cm}|}
\hline
\textbf{ID:} & \textbf{Measure:} & \textbf{Value:} & \textbf{Priority:} & \textbf{Comment:}\\
\hline

NF-14 & Time to restart after failure &
{\textless} 10 min &
HIGH &
\\\hline

NF-15 &
Probability of data corruption on failure &
{\textless} 1\% &
MED & What kind of data\\
\hline

NF-16 & Expected living time & 10-15 years & HIGH & \\
\hline

NF-17 & Execution node & = 1  & HIGH &\\
\hline

NF-18 & Execution nodes & {\textgreater} 1 & MED & It should be possible to
implement more\\
\hline
\end{supertabular}

\subsubsection{Portability/Scalability}

\begin{supertabular}{|l|p{5cm}|p{1.8cm}|l|p{4.4cm}|}
\hline
\textbf{ID:} & \textbf{Measure:} & \textbf{Value:} & \textbf{Priority:} & \textbf{Comment:}\\
\hline

NF-19 & Extensibility && HIGH & adding features, and carry-forward of
customizations at next major version upgrade\\
\hline

NF-20 & Module-based code && HIGH & The code should be easy to maintain\\
\hline
\end{supertabular}

\subsubsection{Other}
\begin{supertabular}{|l|p{5cm}|p{1.8cm}|l|p{4.4cm}|}
\hline
\textbf{ID:} & \textbf{Measure:} & \textbf{Value:} & \textbf{Priority:} & \textbf{Comment:}\\
\hline

NF-21 & Accessibility & Internal \ and external & HIGH & slightly more
important than external \\
\hline

NF-22 & Emotional factors & FUN & HIGH & It should be fun\\
\hline

NF-23 & Open-source & GPL & LOW & \\
\hline
\end{supertabular}

\subsection{Security}
While security requirements are non-functional, we decided to do the
security requirements engineering as a separate process. Table X.X
shows the listing.
Table can be interpreted in the following way:
\begin{itemize}
    \item In ID the first two letters stands for the property of the requirement.
    \item Measure describes how/what the property should measure.
\end{itemize}

\subsubsection{Authentication and Authorization}
\begin{supertabular}{|l|p{7cm}|l|p{5cm}|}
\hline
\textbf{ID} & \textbf{Measure} & \textbf{Priority} & \textbf{Comment} \\ 
\hline

S-01 & No user in any given user group shall be able to perform any operation
outside of the definition of the requirements & MED & \\ 
\hline

S-02 & An authenticated user shall not be able to perform any operation,
asanother user & & \\ 
\hline

S-03 & After an authenticated user performs an action to be logged out,
thatuser will need to log in to re-authenticate & & E.g. \ session-cookies
should not remain such that you can still re-login\\ 
\hline

S-04 & No user shall gain administrative rights without manual approval
ofcurrent administrators & & Ensure no user is registered as admin by mistake,
no scripts thatautomatically escalates privileges to administrator when
conditions aremet\\ 
\hline

S-05 & aprovided interface and provide mandatory credentials. & &
Sometimesusers a identified by session ID's\\ 
\hline

S-06 & To authorize, you will either need to provide mandatory usercredentials
through an interface, or have a valid session ID. & & \\ 
\hline

S-07 & Session tokens shall be unique to one computer only & & Not possible to
simply acquire a session ID and use it on other computers toauthenticate\\
\hline

\end{supertabular}

\subsubsection{Immunity}
\begin{supertabular}{|l|p{7cm}|l|p{5cm}|}
\hline
\textbf{ID} & \textbf{Measure} & \textbf{Priority} & \textbf{Comment} \\ 
\hline
S-08 & No front-end exposed input-fields shall besusceptible to injection
attacks & & \\ 
\hline

S-09 & All data that passes the trust zone shall be in plaintext, andvalidated
for injection attacks & & \\ 
\hline

S-10 & Except from contest program submissions, data from non-developers
canonly be directed/saved in databases. & & E.g.\ you shall not be able to
create files in the source directory.\\ 
\hline

S-11 & Uploaded file scripts shall not write to any file & & \\ 
\hline

S-12 & Uploaded file scripts shall not read from any other file than stdin && \\ 
\hline

S-13 & Uploaded file scripts shall not access network, threading, or anyother
external service not needed to solve a problem.  & & \\ 
\hline

S-14 & Data from a user shall not be modified & & \\ 
\hline
\end{supertabular}


\subsubsection{Non-repudiation}
\begin{supertabular}{|l|p{7cm}|l|p{5cm}|}
\hline
\textbf{ID} & \textbf{Measure} & \textbf{Priority} & \textbf{Comment} \\ 
\hline
S-15 & All modifications of data shall be logged & &\\ 
\hline

S-16 & All log entries shall contain username(s) and a timestamp with dayand
current hour & & \\ 
\hline

S-17 & Logs will be backed up & & \\
\hline

S-18 & A team's score shall not be affected by anything other thanwhat is given
in the contest rules & & \\ 
\hline
\end{supertabular}

\newpage
\subsubsection{Privacy}
\begin{supertabular}{|l|p{7cm}|l|p{5cm}|}
\hline
\textbf{ID} & \textbf{Measure} & \textbf{Priority} & \textbf{Comment} \\ 
\hline

S-19 & Sensitive user data shall not be stored in plaintext  & & \\ 
\hline

S-20 & Every user-field that is stored shall be justified in therequirements
specification & & \\ 
\hline

S-21 & No sensitive data shall be exposed publicly, even if it is encrypted& &
\\ 
\hline

S-22 & User-data for a given user shall not be modified without that user's
consent.  & & \\ 
\hline
\end{supertabular}

\subsubsection{Auditing}
\begin{supertabular}{|l|p{7cm}|l|p{5cm}|}
\hline
\textbf{ID} & \textbf{Measure} & \textbf{Priority} & \textbf{Comment} \\ 
\hline

S-23 & Database shall be manually/automatically checked/verified for
inconsistency or errors before an event. & & \\
\hline

S-24 & Password that are used in development shall not bepublicly available
& &\\
\hline
\end{supertabular}

\subsection{Requirements Not Met}
We met most of the requirements in time to their respective milestone,
but there were some minor requirements we had to drop because of time
constraints. All of them were priority LOW.\ Here are the requirements
we did not complete:

\begin{supertabular}{|m{5.0566597in}|m{1.2858598in}|}
\hline
A judge shall get a notification when received a
question &
FJ-09\\\hline
A functionary shall be able to register a balloon
colour to each task/problem &
FF-01\\\hline
The system shall be able to gather a large
variety of statistic specified by the admin &
FO-02\\\hline
\end{supertabular}

The reason they were not completed was due to the their low priority and
time constraint. In addition to the unfinished requirements we also ended up with some
requirements that we technically did meet, but not in an ideal way.
This was in agreement with the customer. These are the partially met
requirements:

\begin{supertabular}{|l|l|}
    \hline
    An admin can add a node & FA-10 \\
    \hline
    An admin can remove a node & FA-11 \\
    \hline
    An admin can manage a node.  & FA-12 \\
    \hline
    An admin can add more than one node & FA-13 \\
    \hline
    Response from action & NF-01 \\
    \hline
    Logs will be backed up & NR-03 \\
    \hline
\end{supertabular}

Unfortunately an admin can only manage the execution nodes through the
code. This is planned to be fixed before the next contest. The response
time did unfortunately exceed 1.5 seconds during the contest. This was
due to a bad implementation choice, described in~\ref{implementation}. NR-03
had to be overruled during the contest.  This is discussed in detail in section
\textit{development}.
