\chapter{Installation Guide}
This is the complete installation guide for GentleIDI.\ The guide will assume
that the reader has got some basic Linux skills. You should be
capable of installing packages by means of a package-manager like apt,
yum etc.

Though GentleIDI is not tightly linked with any specific linux distro,
this guide assumes that you're using Ubuntu Server
14.04. This is the only distro on which the system has been tested
thoroughly at the time of writing.

GentleIDI is in many ways a straightforward Django-based website, and
hence there are a lot of possible setups to choose from. This guide is
inspired by a guide written by Michal Karzynski\footnote{\url{http://michal.karzynski.pl/blog/2013/06/09/django-nginx-gunicorn-virtualenv-supervisor/}}, and will guide you
through the steps of setting up the system using a combination of
Gunicorn and Nginx.

\section{Creating Your Users}
Running a website as a user with root privileges or anything of the sort
is far from recommended. Therefore you are advised to create a new
user and a new usergroup. The names of both the group and the user can
be chosen as you please, but the rest of the guide will stick to using
a user called gentleidi and a group named webapps.

\begin{verbatim}
sudo mkdir -p /webapps/gentleidi
sudo groupadd --system webapps
sudo useradd --system --gid webapps --home /webapps/gentleidi gentleidi
sudo chown gentleidi:webapps /webapps/gentleidi/
\end{verbatim}

Now you have a user named gentleidi which is a member of the usergroup
webapps, and whose home directory is /webapps/gentleidi.

In addition to the user we just created, we need another user,
specifically used to run the untrusted software submitted by the
contestants. GentleIDI assumes that this user is named gentlemember.
However, changing this value in the source is no complicated matter.

\begin{verbatim}
sudo useradd --system gentlemember
\end{verbatim}

The system needs to be able to execute commands both as gentleidi and
gentlemember. As the Web server runs as gentleidi we need to make sure
that gentleidi can execute commands as gentlemember. Add the following
line to your sudoers file.\

\begin{verbatim}
gentleidi ALL=(gentlemember) NOPASSWD:ALL
\end{verbatim}

If you don't know how to edit your sudoers, to open the
sudoers file in a text editor simply type the following command:

\begin{verbatim}
sudo visudo
\end{verbatim}

Now we've got two users, one capable of executing
commands as the other. What we want to do now is to ensure that
gentlemember is unable to communicate via network. This is done by
applying two rather straightforward iptable rules.

\begin{verbatim}
sudo iptables -A OUTPUT -m owner --uid-owner gentlemember -j LOG
sudo iptables -A OUTPUT -m owner --uid-owner gentlemember -j REJECT
\end{verbatim}

Though this will restrict the user's network access, be
aware of software installed on your system which is capable of
switching to another user.

\section{Setting Up the Environment}

Due to a lot of strict changes made in Python versions, a lot of
libraries do not work across different versions of Python. This leaves
Python in a situation where program A might need Python to be version X
and program B might need python to be version Y. To solve this problem 
you can set up a virtual environment.

Virtual environments is a way of setting up separate python setups for
different sets of programs.

What we want to do is to turn the home directory of the gentleidi user
into a virtual environment.

\begin{verbatim}
sudo apt-get install python-virtualenv
sudo su gentleidi
virtualenv /webapps/gentleidi/env
\end{verbatim}

Now that you've got a virtual environment you can start
filling it with something useful, like the content of the
project's Git repository.

\begin{verbatim}
cp -r /path/to/repo/IDIOpen/ /webapps/gentleidi/
\end{verbatim}

Please note that you only need the wsgi folder from the repository, however,
updating is a lot easier when all you've got to do is
pull the latest version directly using Git. The downside is that you
could possibly end up committing your production system configuration
files etc. to the repo. However, we're going to assume
that you will not be developing directly in your production system, and
thereby avoid the hazard.

Before leaving this step, ensure that the files in /webapps/gentleidi
has got the correct file permissions.

\begin{verbatim}
sudo chown -R gentleidi:webapps /webapps/gentleidi
\end{verbatim}

\section{Installing Required Packages}

Now it's time to start making sure that
you've got the packages you need to run GentleIDI.

\begin{verbatim}
sudo apt-get install git nginx libmysqlclient-dev python-dev
\end{verbatim}

You might already have most of these packages, however, better safe than
sorry.

The next thing you need to do before continuing is to log in as
gentleidi and activate your newly created virtual environment.

\begin{verbatim}
sudo su gentleidi
source /webapps/gentleidi/env/bin/activate
\end{verbatim}

Installing the required Python packages via PyPI is easily done. In the
project root directory there's a file named
requirements.txt. This file is simply a list of required packages, to
install them simply execute the following:

\begin{verbatim}
pip install -r requirements.txt
\end{verbatim}

\section{Database}
GentleIDI needs a database to store its data. This guide will show you
how to setup GentleIDI with a MySQL database server, however, if you
feel like using PostgreSQL, or even SQLite, then please do.
Any database server supported by Django is supported by GentleIDI.

Naturally you don't need to install the database server
on the same host as the Web server, that's what
we'll do for now.

\begin{verbatim}
sudo apt-get install mysql-server
\end{verbatim}

Now what we need to do is to create a database and a MySQL user that
GentleIDI can use. During the install process you were required to set
a root password for the MySQL-server. Login as root and perform the
following commands:

\begin{verbatim}
CREATE USER gentledb'@'localhost' IDENTIFIED BY 'password';
GRANT ALL PRIVILEGES ON * \@. * TO 'newuser'@'localhost';
FLUSH PRIVILEGES;
CREATE DATABASE gentleidi CHARACTER SET uft8 COLLATE utf8_general_ci;
\end{verbatim}

Remember to replace ``gentledb'' and ``password'' with a suitable username and
password. Now you need to ensure that GentleIDI uses your newly created
database. Edit the DATABASES entry in IDIOpen/wsgi/openshift/settings.py

\begin{verbatim}
if MYSQL:
     DATABASES = {
          'default': {
              'ENGINE'   : 'django.db.backends.mysql',
              'NAME'     : 'gentleidi',
              'USER'     : 'gentledb',
              'PASSWORD' : 'password',
              'HOST'     : 'localhost',
              'PORT'     : '3306',
\end{verbatim}

In order to make sure that the database is working properly, log in as
gentleidi, activate your environment and synchronize
GentleIDI's database.

\begin{verbatim}
sudo su gentleidi
source /webapps/gentleidi/env/bin/activate
python /webapps/gentleidi/IDIOpen/wsgi/manage.py syncdb
python /webapps/gentleidi/IDIOpen/wsgi/manage.py migrate
\end{verbatim}

If this command terminates properly, then your database should be good
to go. In fact you should be able to run GentleIDI on a development
server at this point. But first, you need to create an admin account. To
do so, simply execute the following:

\begin{verbatim}
python /webapps/gentleidi/IDIOpen/wsgi/openshift/manage.py createsuperuser
\end{verbatim}

To start the development server run:

\begin{verbatim}
python /webapps/gentleidi/IDIOpen/wsgi/openshift/manage.py runserver
\end{verbatim}

You should now have a working website running on port 8000. However, you
have no execution nodes available to evaluate submissions, and
you're using Django's development
server, which scales horribly.

\section{Gunicorn}
Now it's time to install replace the Django development
server with a proper application server, Gunicorn. Remember to be logged
in as gentleidi, and to activate your environment before proceeding.

\begin{verbatim}
pip install gunicorn
\end{verbatim}

Now we need a script that launches Gunicorn and GentleIDI appropriately.

\begin{verbatim}
#!/bin/bash
# Name of the application
NAME=GentleIDI

DJANGODIR=/webapps/gentleidi/IDIOpen/wsgi/ # Django project directory
SOCKFILE=/webapps/gentleidi/run/gunicorn.sock # we will communicate using this unix socket
USER=gentleidi # the user to run as
GROUP=webapps # the group to run as

NUM_WORKERS=3  # how many worker processes should Gunicorn spawn
DJANGO_SETTINGS_MODULE=openshift.settings # which settings file should Django use
DJANGO_WSGI_MODULE=openshift.wsgi # WSGI module name

echo "Starting NAME as whoami"

# Activate the virtual environment
cd DJANGODIR
source /webapps/gentleidi/env/bin/activate
export DJANGO_SETTINGS_MODULE=$DJANGO_SETTINGS_MODULE
export PYTHONPATH=$DJANGODIR:$PYTHONPATH

# Create the run directory if it doesn't exist
RUNDIR=$(dirname $SOCKFILE)
test -d $RUNDIR {textbar}{textbar} mkdir -p $RUNDIR
# Start your Django Unicorn
# Programs meant to be run under supervisor should not daemonize themselves 
#(do not use --daemon)

exec /webapps/gentleidi/env/bin/gunicorn ${DJANGO_WSGI_MODULE}:application \
 --name $NAME 
 --workers $NUM_WORKERS 
 --user=$USER --group=$GROUP 
 --log-level=debug {textbackslash}
 --bind=unix:$SOCKFILE
\end{verbatim}

Place the contents of the previous page in the following file:
\begin{verbatim}
/webapps/gentleidi/env/bin/gunicorn_start
\end{verbatim}

Make sure that the script is executable:

\begin{verbatim}
sudo chmod u+x /webapps/gentleidi/env/bin/gunicorn_start
\end{verbatim}

\subsection{Nginx}
As mentioned previously this setup relies on a combination of Gunicorn
and Nginx. At this point gunicorn should be working properly, and
it's time to setup Nginx.

If you have not already installed nginx, do so now:

\begin{verbatim}
sudo apt-get install nginx
\end{verbatim}

Now you need to create an nginx configuration file for your Web site, 
in this case the file is called ``gentleidi''.

Store the content found below in the following file: 
\begin{verbatim}
/etc/nginx/sites-available/gentleidi
\end{verbatim}

\begin{verbatim}
upstream hello_app_server {
    server unix:/webapps/gentleidi/run/gunicorn.sock fail_timeout=0;
}
server {
    listen 80;
    servername example.com;
    client_max_body_size 4G;
    access_log /webapps/gentleidi/logs/nginx-access.log;
    error_log /webapps/gentleidi/logs/nginx-error.log;
    location /static/ {
        alias   /webapps/gentleidi/IDIOpen/wsgi/static/;
}
location /media/ {
    alias   /webapps/gentleidi/IDIOpen/wsgi/media/;
}
location / {
    proxy_set_header X-Forwarded-For
    $proxy_add_x_forwarded_for;
       proxy_set_header Host $http_host;

       proxy_redirect off;
       if (!-f $request_filename) {
           proxy_pass http://hello_app_server;
           break;
       }
   }
# Error pages
error_page 500 502 503 504 /500.html;
   location = /500.html {
       root  /webapps/gentleidi/IDIOpen/wsgi/static/;
   }
}
#EOF
\end{verbatim}

In this configuration Nginx is configured to log all accesses and errors. These
log files need to be created with the following commands:

\begin{verbatim}
sudo su gentleidi
mkdir /webapps/gentleidi/logs
touch /webapps/gentleidi/logs/nginx-access.log
touch /webapps/gentleidi/logs/nginx-error
exit
\end{verbatim}

All you need to do at this point is to enable the Nginx site. This is
done simply by creating a symbolic link from the configuration file in
sites-available to sites-enabled.

\begin{verbatim}
sudo ln -s /etc/nginx/sites-available/gentleidi
/etc/nginx/sites-enabled/
sudo rm /etc/nginx/sites-enabled/default
sudo service nginx restart
\end{verbatim}

You should now have a working website. All that is left
is making management a little easier, and adding some execution nodes.

\section{Supervisor}
Supervisor is a utility for defining and managing jobs. In this case
we're going to define two jobs, one for managing the
website, and another for managing an execution node.

You need to create two files to make this happen:

\begin{verbatim}
/etc/supervisor/conf.d/gentleidi.conf

[program:gentleidi]
command = /webapps/gentleidi/env/bin/gunicorn_start
user = gentleidi
stdout_logfile = /webapps/gentleidi/logs/gunicorn_supervisor.log
redirect_stderr = true
\#EOF
\end{verbatim}

\begin{verbatim}
/etc/supervisor/conf.d/celery.conf
[program:celery]
command=/webapps/gentleidi/env/bin/celery worker -A openshift -l info
directory=/webapps/gentleidi/IDIOpen/wsgi
environment=PATH='/webapps/gentleidi/env/bin:%(ENV_PATH)s'
user=gentleidi
autostart=true
autorestart=true
redirect_stderr=True
#EOF
\end{verbatim}

Create the log files that you've referenced.
\begin{lstlisting}[language=bash]
mkdir /webapps/gentleidi/logs/
touch /webapps/gentleidi/logs/gunicorn_supervisor.log
\end{lstlisting}

Read the newly created configuration files.
\begin{lstlisting}[language=bash]
sudo supervisorctl reread
sudo supervisorctl update
sudo supervisorctl restart all
\end{lstlisting}

\section{Multiple Execution Nodes}
The easiest way of setting up multiple execution nodes is to clone the
setup on your Web server to other machines and then making minor
changes.
When setting up multiple execution nodes there are two changes that need to be
made. The directory \begin{verbatim}/webapps/gentleidi/IDIOpen/wsgi/private/submissions\end{verbatim}
needs to be shared between all the execution nodes. How you decide to make this
happen is up to you. However, SSHFS is possibly the easiest solution. Whatever
way you decide to mount the directory on your execution nodes, make sure that
multiple users are allowed to access it, e.g.\ the ``allow\_other''
option for SSHFS.
You also need to make sure that all your execution nodes have access to the
same database. Make sure that the settings.py is not set to localhost, but
rather points to whatever host you decide to use as a database server. Some
configuration of your database server might be needed in order for it to accept
remote connections. MySQL servers need to change the bind-address property in
the /etc/mysql/my.cnf to their actual IP, not localhost(127.0.0.1).
You also need to change the grants for the MySQL user in such a way that it is
allowed to connect remotely to the database.
