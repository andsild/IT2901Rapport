\chapter{Introduction}
\section{The course}
Our group and assignment has been delegated as part of the course
IT2901: ``Informatics Project II''
at NTNU.\ The work covers 15 course credits, equivalent to a 50\% work
position for one academic semester. IT2901 is offered only to those
that are enrolled on the NTNU's informatics BSc
programme.

The primary purpose of the course is to let students apply their
knowledge\ from other courses. This is rendered through a project for a
real customer. The students have to communicate independently with
their client, and deliver a software product that answers the
client's needs. 

Grades are based on the satisfaction of the customers and an evaluation
of the development process. The latter will be reviewed through written
reports and timesheets, as provided in this document. Furthermore, it
is important that students have met the given deadlines and documented
their work in a structured manner.

\section{The Group}

The team consists of six members. All the members of the group are
completing their BSc degree in Computer Science from NTNU in 2014. We
had prior experience working together, and knew each other well. With
many shared courses and similar interests, the team are all at a
somewhat similar level of competence. However, we have different areas
of expertise, and exploiting this has been a key to success on previous
occasions. For a detailed description of each member, see the listing
below.

\subsubsection*{Anders Sildnes}
Throughout his BSc, Anders has been taking courses related to algorithms
and program security. Apart from his studies, he is developing for
Engineers without Borders NTNU and spending time with open-source
projects and other Linux tools.

\subsubsection{Eirik Fosse}
Eirik has a primary interest in artificial intelligence and machine
learning. In the course of his bachelor's degree
he's focused on programming, mathematics, and
evolutionary simulation.

\subsubsection{Filip Fjuk Egge}
While achieving his degree, Filip has taken courses focused on a path
related to system development and security. He has a varied education
and knowledge\ on different aspects of computer science. 

\subsubsection{Haakon Konrad William Aasebø}
Haakon has selected disciplines related to mathematics and algorithms.
Apart from being a student at NTNU he is playing football at NTNUI in
the third division. 

\subsubsection{Håkon Gimnes Kaurel}
During his time at NTNU, Håkon has been keeping a primary focus on
courses related to programming and the intersection between hardware and
software. He also has experience as an app developer, and has extensive
knowledge of the GNU/Linux operating system. 

\subsubsection{Tino Lazreg}
Tino has been taking courses related to different aspects of software
engineering, like programming, system architecture, human-machine
interaction. Besides doing a BSc, Tino also works as a student
assistant in a human-machine interaction course on NTNU. 

\section{The Customer}
Our customer is IDI Open. They are responsible for the annual
programming contest mentioned in 1.2. Christian Chavez is our main contact for
the project, but his two colleagues, Christian Neverdal Jonassen and Finn
Inderhaug Holme, were
also available for questions. They are all students of computer science
at NTNU. 

\section{The Contest}
IDI Open is a programming contest where teams of up to three people meet
and solve programming problems of various difficulty. The contest lasts
five hours, and the objective is to solve as many problems as possible.
The contest is open for all types of programmers, from students of all
grades to professors and other professionals from the IT industry.
Various prizes are given to the teams based on their performance. There
are usually 8-12 problems in a contest. To make the competition fun for
everyone, there are typically some problems that are easy enough even
for novice programmers to handle. The main objective is to solve the
highest amount of problems in the shortest amount of time.

\section{Stakeholders}
Our stakeholder fall into two different categories: the ones involved in
the course, and those involved in the product and competition.

\subsection{Course}
\begin{description}
\item[Supervisor]
The supervisor's job consists of guiding and helping us
through this project. This aid was primarily focused on the development
process and the writing of this report. The supervisor tries to ensure
that the developers communicate properly and have a structured approach
to developing the end product. To verify this, we have had biweekly
status reports delivered to the supervisor, as well as regular
meetings.

\item[Examiner]
The examiner(s) is responsible for determining our final grade. Unlike
the other stakeholders, we have not communicated with the examiner
throughout the development process. Though, the examiner has got access
to all the documents the supervisor has got access to.
\end{description}

\subsection{Product and Competition}
\begin{description}
\item[IDI Open]
The project's primary stakeholders. They are the host of
the competition in which our product was used. Their inclusion in this
product comprised all aspects of our project.

\item[Judges]
The judges are hired by IDI Open to supervise the competition, service
contestants and create problem sets. Throughout the process they have given
feedback to our customers, IDI Open, about our product. Naturally, the
judges are important to the contest, so it is important that they are
satisfied with the software they have to use.

\item[Developers]
The developers are responsible for satisfying all other parties. Similar
to the customer, our involvement in this project is total.

\item[Maintainers]
As IDI Open is an annual event, our end product, will be used for many 
years in the future. At a point, we assume the code will need to be 
extended or modified by another developer team. 
As such, the quality of our product will impact them.

\item[Sponsors]
Each contest has companies sponsoring them. In exchange for money and
services, the sponsors get exposure through ads on the website and are
given the opportunity to hold a short presentation after the contest. Naturally, the
sponsors want to associate their name with a successful product.
Therefore, the sponsors rely on successful contests. This is
heavily based on our products performance.

\item[Contestants]
The actions of contestants are all through our software; our product
will be their medium to take part in IDI Open. Reliability and
usability is key to keep the contestants happy. The contestants also
gave feedback to the customers about their user experience. Thus, how
satisfied the contestants are impacts the developer's
evaluation.
\end{description}

\section{Goals}
Our assignment is to replace the existing system
used in IDI Open. We were given sole responsibility for our project; no
other team or organization of developers has had responsibility for our
solution. This gave us inspiration to do the best we could, and to give
the customer something both we and they could be proud of for years to come. 
If the product is good enough it would hopefully
also be used in larger programming competitions, maybe even
international ones.

