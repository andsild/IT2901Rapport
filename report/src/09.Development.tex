
\chapter{Development}

This document describes the different phase of development the group
went through in order to finish the product. To increase readability
the first part of the document describe the process working towards the
milestones, as can be viewed in fig 3.3. The second part will describe
each sprint in more detail including work done/completed. 


\bigskip

\section{Towards the milestones}


\bigskip

\subsection{Milestone M-01 - Preliminary report.}

From start to 09.02

Eager to start, we had our first meeting 15.01.2014. During this meeting
we discussed which task we wanted apply to. We decided on
{\textquotedblleft}IDI\_ProgrammingContest{\textquotedblright}, which
we received. 


\bigskip

After receiving the project assignment, we discussed our ambitions for
the course and the end product. We agreed that we had a shared goal to
receive a top grade in this course, and we that we would resolve in
whatever means necessary to achieve this goal. The group was in doubt
if we should try popular, enterprise-level tools and frameworks, or if
we should stick to basic, previously used tools. We decided to let each
member of the group to explore a tool on his own and present his
experience to the others. If the tool seemed usable, we incorporated it
into our project.


\bigskip

Our primary concern was that we would spend time on suboptimal tools,
methods or frameworks. Thus, the group spent much time discussing and
modeling the application to come. 


\bigskip

\subsection{Milstone M-02 - Mid-semester report}

From 09.02 to 09.03

Being aware of the large amount of programming ahead of us, we aimed to
have the mid-semester report finished one week before the actual
deadline.


\bigskip

To shorten meeting time and strengthen our task overview, we had a
meeting thoroughly discussing how SCRUM worked. We decided to adhere
more of the conventional SCRUM standard. As a consequence we started to
draft release- and backlog. This resulted in a reduction in the number
of hours used to administering and task delegations. We also got a
better overview of the final application. This meant that we could
reduce the amount of models and focus more on the code. 


\bigskip

The mid-semester report finished as planned one week before our
deadline. We more or less completed our testing plans and concluded on
management structure. The biggest challenge was how to implement
support for user handling. 


\bigskip

\subsection{Milestone M-03 - First release}

From 09.02 to 19.03

Having finished the midterm report, the group now had a structured
overview of the requirements specification and approach for
development. We had much coding to do in order to reach the third
milestone. We tried to agree on an optimal approach, but concluded that
we had to {\textquotedblleft}just get started{\textquotedblright}. In
our sprint backlogs the amount of coding assignments grew. To induce
more coding, we arranged informal coding nights in order to trigger
{\textquotedblleft}learning by doing{\textquotedblright} and improved
our progression.


\bigskip

By the time we had finished the necessary pre-studies and requirements
we already had some functionality. However, there were still work
remaining, as suggested by our work breakdown structure. In addition we
had a meeting with the customer where they proposed some new
requirements, and reprioritized a few other. 


\bigskip

In advance to the first release we had some meetings with the customer.
We were a little nervous regarding some of the design choices. The
meeting discussing the design went well. We had formerly agreed on our
mock up-design. There were however a few discrepancies between the
delivery and what the customer wanted. \ 


\bigskip

Our first deadline with the customer 19.03, but the actual release of
the website was delayed to after the weekend, for external reasons. 


\bigskip

\subsection{Milestone M.04 - Presentation}

From 09.02 to 19.03

Since the presentation was scheduled at the same time as our first
release we did not have a time to prepare much for this presentations.
Nevertheless, we received valuable feedback from other groups.


\bigskip

\subsection{Milestone M-05 - Beta Release}

From 19.03 to 11.04

Working toward the beta release was challenging. Increasingly, we
experienced that modeling the application before coding was not an
optimal solution. Thus, we began to code without relying on diagrams to
aid us. We sustained this approach until the end of the project.


\bigskip

With limited time, it became necessary to prioritize some tasks over
others. Our improved product backlog, created earlier, proved to be a
big benefit. As mentioned previously, we felt that it was hard to
predict the outcome of the development process, so we decided not to
update the gantt-diagram. Instead we relied on our own options and
customer prioritizations This was due to our new understanding of what
needed to be completed when.


\bigskip

We did make some progress with our development, but still had some
aspects of our frameworks that needed to be learned. As the weeks went
by we increased our work estimates and grew more familiar with the
framework. Still our models seldom related to the actual end result. It
was not something we felt was a big problem, we did make progress. \ 


\bigskip


\bigskip

\subsection{Milestone M-06 - IDIOpen test event}

From 11.04 to 26.04

We still had quite a few packages to implement and we were uncertain
about how long time we had to spend on each of them. As a consequence
we had to shorten our easter vacation.Spending a lot of time together,
every day for weeks, may cause tension in groups. We felt it was
important to create an environment to ease the tensions.Therefore we
took breaks from the coding, eating pizza and playing foosball. 


\bigskip

We started every day discussing what we were suppose to do, similar to a
daily scrum. We believed all members had a good tacit understanding of
what needed to be done, so we transitioned from sprint backlogs to
daily TODO lists. These lists were written informally for the sake of
brevity.\newline
\newline
The days were long, lasting from 09:00 to 24:00. Packages were
implemented at a high pace and we started to become aware of the final
product. The biggest challenges were to get the execution node up and
running, highscore table and content management for the judges. Testing
was also completed. We also had sufficient time to implement some of
the lower prioritized requirements. 


\bigskip

During the test event, we sat at our own table and received feedback
from the judges and volunteers that had shown up. The fact that some of
the judges were considered really good programmers made us a little
nervous. They did give us feedback and a list of new requirements to be
implemented. These were minor fixes, mostly related to the user
interface. The test event itself was considered a success: all the
judges approved our system.


\bigskip

\subsection{Milestone M-07 - IDI Open}

From 26.04 to 03.05

After the test event we got a new list of requirements. There was only
one week to the actual event and we had to carefully pick those we and
the customer felt were the most important. We implemented support for
several execution nodes, refined the content management and fixed small
bugs. Some tasks were complex, so it was a challenge to analyze if we
would be able to finish them on time. The most advanced tasks we were
given after the test event was that the judges wanted a better overview
over the contest. More precisely, they wanted access to the whole
competition, and all the functionality, before the contest started. The
customer also wanted to be able to export data to CSV and LaTeX. This
task did, at first, seem lightweight, but turned out to be much more
extensive. While finishing on time, this consumed more hours than
initially planned.


\bigskip

In total there were 92 teams taking part in IDIOpen14, and a total of
214 registered users in the system. When the contest officially started
and the problem sets were released. All users simultaneously accessed
the same resource caused a spike on the system load. We were been told
by our customer that the old system had previously buckled under the
pressure from the spike at the beginning of the contest. Our system
did, however, handle this well. Thus, the start of the contest went
well. 


\bigskip

At one point the system went down for a few minutes. This was because we
ran out of hard disk space on our main server. In other words, the
system had nowhere to store its data, and was unable to handle the
requests made by users. After a couple of minutes of deleting
unnecessary files we discovered that for every file that we removed we
only bought ourselves a couple of more minutes of uptime. Somewhere in
the file system there was a file growing at an alarming pace. To
identify this file was challenge. By monitoring the
server{\textquoteright}s processes we found that the application
responsible for load-balancing was logging every event under
{\textquotedblleft}debug{\textquotedblright} setting. This meant that
every request to the webserver caused an extensive log entry. This
resulted in a 1MB/S disk write rate. The rate was small enough so that
we could easily monitor and periodically erase the logs to clear out
disk space. We could have disabled logging, however, that would have
required a restart of the database server and thereby downtime. 


\bigskip

After this problem was resolved the rest of the contest went without any
significant issues. At our peak system load we handled a total of 12
concurrent submissions which was more than enough. The website was
responsive and working properly. Except the highscore list, which we
knew had performance issue. These issues did not have a significant
impact for the user experience.


\bigskip

\subsection{Milestone M-08 - Final report}

from 03.05 to 30.05

After the final event we were all exhausted. The following week we only
did some administrative tasks. We based ourselves on the mid-semester
report and the feedback we got from supervisor and external sources. 


\bigskip

Sprint by sprint

We have documented each sprint. These are given in appendix X. An
example is given here in table X.X.\newline

