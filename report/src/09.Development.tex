\chapter{Development}

This document describes the different phases of development the group
went through in order to finish the product. To increase readability
the first part of the document describes the process of working towards the
milestones, as can be viewed in figure~\ref{table:milestones}. The second part
describes each sprint in more detail including work done/completed. 

\section{Working Towards the Milestones}
\subsection{Milestone M-01 - Preliminary Report}
\label{sec:M01}

From start to 09.02.2014

Eager to start, we had our first meeting 15.01.2014. During this meeting we
discussed which tasks we wanted apply for.  After receiving the project
assignment, we discussed our ambitions for the course and the end product. We
agreed that we had a shared goal to receive a top grade in this course, and
that we where all prepared to put in the work required to achieve this goal.
The group was in doubt if we should try popular, enterprise-level tools and
frameworks, or if we should stick to basic, previously used tools. We decided
to let each member of the group to explore a tool on his own and present his
experience to the others. If the tool seemed usable, we incorporated it into
our project.

Our primary concern was that we would spend time on suboptimal tools, methods
or frameworks. Thus, the group spent much time discussing and modeling the
application to come. 

\subsection{Milstone M-02 - Mid-semester Report}
\label{sec:M02}
From 09.02.2014 to 09.03.2014

Being aware of the large amount of programming ahead of us, we aimed to have
the mid-semester report finished one week before the actual deadline. To
shorten meeting time and strengthen our task overview, we had a meeting
thoroughly discussing how Scrum worked. We decided to adhere more of the
conventional Scrum standard. As a consequence we started to draft release and
product backlogs. This resulted in a reduction in the number of hours used to
administer and delegate tasks. We also got a better overview of what we wanted
the end product to look like. This meant that we could reduce the amount of
modeling, and focus more on the code. 

The mid-semester report finished as planned one week before our deadline. We
more or less completed our testing plans and concluded on management structure.
The biggest challenge was how to implement support for user handling. 

\subsection{Milestone M-03 - First Release}
\label{sec:M03}
From 09.02.2014 to 19.03.2014

Having finished the mid-semester report, the group now had a structured
overview of the requirements specification, and approach to development. We had
much coding to do in order to reach the third milestone. We tried to agree on
an optimal approach, but concluded that we had to ``just get started''. In our
sprint backlogs the amount of coding assignments grew. To induce more coding,
we arranged informal coding nights in order to trigger ``learning by doing''
and improved our progression.

By the time we had finished the necessary prestudies and requirements, we
already had some functionality. However, there was still work remaining, as
suggested by our work breakdown structure. In addition we had a meeting with
the customer where they proposed some new requirements, and reprioritized a few
others. 

In advance to the first release we had some meetings with the customer.  We
were a little nervous regarding some of the design choices, however, the
meeting discussing the design went well. We had formerly agreed on our mock
up-design, although there were a few discrepancies between the delivery and
what the customer wanted.

The deadline for our first delivery to the customer was 19.03.2014, but the
actual release of the website was delayed to after the weekend, for external
reasons. 

\subsection{Milestone M.04 - Presentation}
\label{sec:M04}
From 09.02.2014 to 19.03.2014

Since the presentation was scheduled at the same time as our first
release, we did not have time to prepare for this presentation.
Nevertheless, we received valuable feedback from other groups.

\subsection{Milestone M-05 - Beta Release}
\label{sec:M05}
From 19.03 to 11.04

Working toward the beta release was challenging. Increasingly, we
experienced that modeling the application before coding was not an
optimal solution. Thus, we began to code without relying on diagrams to
aid us. We sustained this approach until the end of the project.

With limited time, it became necessary to prioritize some tasks over
others. Our improved product backlog proved to be a
big benefit. As mentioned previously, we felt that it was hard to
predict the outcome of the development process, so we decided not to
update the Gantt diagram. Instead we relied on our own options and
customer prioritizations. This was due to our new understanding of what
needed to be completed when.

We did make some progress with our development, but still had some
aspects of our frameworks that needed to be researched. As the weeks went
by, we increased our work estimates and grew more familiar with the
framework. Still our models seldom related to the actual end result. It
was not something we felt was a big problem, as we where making progress.

\subsection{Milestone M-06 - IDI Open Test Event}
\label{sec:M06}
From 11.04.2014 to 26.04.2014

We still had quite a few packages to implement, and we were uncertain
how much time we needed to spend on each of them. As a consequence
we had to shorten our easter vacation. Spending this much time together,
every day for weeks, may cause tension in groups. We felt it was
important to create an environment to ease the tensions. Therefore we
took breaks from the coding, eating pizza and playing foosball. We started every day discussing what we were suppose to do, similar to a
daily scrum. We believed all members had a good tacit understanding of
what needed to be done, so we transitioned from sprint backlogs to
daily TODO lists. These lists were written informally for the sake of
brevity.\newline
\newline
The days were long, lasting from 09:00 to 24:00. Packages were
implemented at a high pace, and the pieces where finaly starting to fall into place. 
The biggest challenges were to get the execution node up and
running, highscore table, and contest management for the judges. Testing
was also completed. We also had sufficient time to implement some of
the lower prioritized requirements. 

During the test event, we sat at our own table and received feedback
from the judges and volunteers that had shown up. The fact that some of
the judges were considered really good programmers made us a little
nervous. They did give us feedback and a list of new requirements to be
implemented. These were minor fixes, mostly related to the user
interface. The test event itself was considered a success: all the
judges approved our system.

\subsection{Milestone M-07 - IDI Open}
\label{sec:M07}
From 26.04.2014 to 03.05.2014

After the test event we got a new list of requirements. There was only
one week to the actual event, and we had to carefully pick those we and
the customer felt were the most important. We implemented support for
several execution nodes, refined the contest management, and fixed small
bugs. Some tasks were complex, so it was a challenging to predict if we
would be able to finish them on time. The most advanced task we were
given after the test event, was that the judges wanted a better overview
of the contest. I.e. they wanted access to the whole
competition and all the functionality, before the contest started. The
customer also wanted to be able to export data to CSV and LaTeX. This
task seemed lightweight at first, but turned out to be much more
extensive. While finishing on time, this consumed more hours than
initially planned.

In total there were 92 teams taking part in IDI Open 14, and a total of
214 registered users in the system. When the contest officially started
and the problem set was released, all users simultaneously accessed
the same resource. This caused a spike on the system load. We had been told
by our customer that the old system had previously buckled under the
pressure from this spike. Our system did, however, handle this well. 
Thus, the start of the contest went well. 

At one point the system went down for a few minutes. This was because we
ran out of hard disk space on our main server. In other words, the
system had nowhere to store its data, and was unable to handle the
requests made by users. After a couple of minutes of deleting
unnecessary files, we discovered that for every file that we removed, we
only bought ourselves a couple of more minutes of uptime. Somewhere in
the file system there was a file growing at an alarming pace. 
Identifying this file was challenge. By monitoring the
server's processes we found that the database was logging extensively. 
This resulted in a 1MB/s disk write rate. The rate was small enough that
we could easily monitor and periodically erase the log to clear out
disk space. We could have disabled logging, however, that would have
required a restart of the database server and thereby downtime. 

After this problem was resolved the rest of the contest went without any
significant issues. Our system where capable of handeling a total of 12
concurrent submissions, which was more than enough. All parts of the website where
responsive and working properly, except the highscore list, which we
knew had performance issues. These issues did not have a significant
impact on the user experience.

\subsection{Milestone M-08 - Final report}
\label{sec:M08}
from 03.05.2014 to 30.05.2014

After the final event we were all exhausted. The following week we only
did some administrative tasks. We started working on the report based on 
the feedback we got from the supervisor and external sources. 

\subsubsection{Sprint by sprint}
We have documented each sprint. These are given in appendix
~\ref{chap:sprints}. An example is given here in table X.X.
