\section{Design}
This document contains the choices made regarding the process of
designing the front-end of the application, for a more technical
approach see \textit{System Architecture chapter 6}.

Design process

The user interface provided by the previous IDI Open system consisted of
a simple web interface for reading news items, registering teams for
contests, and delivering submissions. GentleIDI is intended to provide
more functionality through its web interface, including but not limited
to change email(requirement FC-02), \ supervisor(requirement FJ-11) and
user management (requirements FC-01, FC-03 and FC-04). As a consequence
we had two options available: reusing and extending the existing
interface design, or creating our own design from scratch.

\begin{figure}[h!]
	\includegraphics[width=6.278in,heigh\t=3.7638in]{./src/figure/a07Design-img1.png} 
	\caption{FIGURECAPTION}
\end{figure}
Fig 7.1
We chose to create our own design from scratch, while still trying to
keep a similar placement of elements from the previous design.The
customer expressed concern regarding how contestants would react to the
transition from the old interface to the new one. With this in mind we
started to create mockups modelling core elements of the website. Our
initial drafts consisted of simple rearrangements of elements found in
the old web interface.

\begin{figure}[h!]
	\includegraphics[width=6.278in,heigh\t=1.7362in]{./src/figure/a07Design-img2.png} 
	\caption{FIGURECAPTION}
\end{figure}

Beyond our three initial mockups we tried a couple of {\textquotedbl}out
of the box{\textquotedbl} approaches to our designs, but none of them
met our standard and was rejected for either being too time-consuming
to implement or too far from what our customer wanted. We had a meeting
with our customer, where we showed our mockups, and what are thoughts
on design had been so far. We wanted to make sure that the customers
was on the same page as us, and that we were not moving beyond the
scope of the project. Our customers wasn't very focused
on the design aspect, but one demand they had was that they wanted the
new site to have the same structure as the old one. One example of what
this means is that the customer wanted us to keep the menu on the
left-side as you can see that the old system has in Fig 7.1. We agreed,
because ge\tting used to a new website can take time, so keeping the
structure similar would ease the transition for our users. With this in
mind we decided to go for one of our initial mockups, the rightmost one
in Fig 7.2, because it had the same structure as the old page, and we
personally favoured that design. As a result, most of the elements
found in the old interface can be found in the new one, and the
transition between using the two is reduced to a minimum.

The task had to be completed in time for milestone M-03, so our main
concern was designing for the functionality needed for that particular
milestone. However, we also had mockups for functionality outside of
this milestone. After milestone M-03 was done, we introduced new design
for new functionality through continuous work on top of a template.

The majority of the front end is stylized using bootstrap[Link til
kilde] as a framework, enabling us to create a site which is both
highly maintainable and aesthetically pleasing at the same time. The
admin interface was created using django-admin-interface with Grappelli
as a skin to give it a modern look. This worked more or less
automatically.


The final page looked like this:
 \begin{figure}[h!]
	\includegraphics[width=6.4272in,heigh\t=3.2602in]{./src/figure/a07Design-img3.png} 
	\caption{FIGURECAPTION}
\end{figure}

The ``black'' frame was in our
initial page coloured blue, but was changed one week before M-07,
idiopen [ REMARK: may be altered ].This illustrates the strongest
functionality of the design, namely customization. It is possible, by
only uploading a new CSS file, to change the whole feel of the website
and give every contest its own theme. The change on IDI OPEN 14, from
blue to black, was done as a consequence of a logo change by Richard
Eide, one of IDI Open's facilitators. The old color
scheme can be viewed in appendix [insert which appendix]. By comparing
fig 7.1 and fig 7.3, you can see that we kept the same structure, but
still made some significant changes to the design.


User interface
The user interface is designed by using a base template. The template is
the same for every part of the webpage, and contains a content block
that changes while you navigate through the different parts. This makes
it easier to add new content to the user interface, because you already
have the base, and don't need to worry about the
header, footer or the menu. We wanted to make it easy for future
developers to take over GentleIDI after us, and therefore we focused on
a versatile user interface, in case they want to add new functionality.

The menu is placed to the left, coping with the western norm stating
that eye placement is natural to the
left\footnote{\ http://research.microsoft.com/en-us/um/people/cutrell/chi09-buschercutrellmorris-eyetrackingforwebsalience.pdf}.
We designed the menu to be versatile. Admins can choose what they want
to show in the menu, except for \textit{Register user} and
\textit{Register team }that are
``hardcoded'' on request from the
customer. This was highly prioritized by our customers, they wanted to
be able to make changes without having to change the code. As mentioned
in Design process 7.1, we designed the user interface after a principle
of versatility. Admins can also change the logo, the sponsor images and
the contact information in the footer.

Buttons, images and icons were surrounded with boxes, for example the
sponsors and the menu buttons, to show that they are different
elements.. There is also one big box surrounding a group of elements,
for example the sponsors. This is consistent with the ge\stalt law of
proximity, that constitutes that humans will naturally group objects
that are close to each other, and view them as a distinct.
This helps the user quickly understand the user interface.


 \begin{figure}[h!]
	\includegraphics[width=2.1807in,heigh\t=0.5972in]{./src/figure/a07Design-img4.png} 
	\caption{FIGURECAPTION}
\end{figure}
\begin{figure}[h!]
	\includegraphics[width=1.0693in,heigh\t=0.5in]{./src/figure/a07Design-img5.png} 
	\caption{FIGURECAPTION}
\end{figure}
\begin{figure}[h!]
	\includegraphics[width=1.1528in,heigh\t=0.4028in]{./src/figure/a07Design-img6.png} 
	\caption{FIGURECAPTION}
\end{figure}
fig 7.4

``To strive for consistency'' is the
first of Shneiderman's eigh\t golden rules of interface
design\footnote{\ https://www.cs.umd.edu/users/ben/goldenrules.html},
and we tried to follow this while making design decisions. As can be
seen in fig 7.4, we decided to use colours that represents the action
each button is connected to. The red button marks that pressing this
will have permanent consequences. We added a textbox prompt that the
user has to answer after pressing a red button, that constitutes to
Schneiderman's fifth and sixth rule, for easy reversal
of actions and error handling. This wasn't added
initially, but we noticed while testing the system that without a
prompt, it could be possible to leave your team by mistake.
 \begin{figure}[h!]
	\includegraphics[width=6.5in,heigh\t=2.5417in]{./src/figure/a07Design-img7.png} 
	\caption{FIGURECAPTION}
\end{figure}
fig 7.5

For the contest page, fig 7.5, we wanted to give the contestant a good
overview of all the problems, their submissions to them, last feedback,
if they solved the problem and the score. It is important to not bury
information to deep in a website. It could be challengi\ng to balance
this while trying not to overload the page with too much information.
We had this in mind when designing this page. We go\t valuable feedback
from the customer concerning what they wanted to be present on the
contest page. They wanted it to be easy for the contestants to access
everything they need, during the competition, through the contest page.
After feedback from the customer, we added links to the clarification
page and highscore table on the contest page. This lowers the
short-term memory load on the contestants, which is consistent with
Shneiderman's eigh\t rule, because they will have
everything accessible on the same page.


Admin interface
 \begin{figure}[h!]
	\includegraphics[width=6.5in,heigh\t=3.0555in]{./src/figure/a07Design-img8.png} 
	\caption{FIGURECAPTION}
\end{figure}

The admin interface is developed as an extension
Django's admin interface. Django comes with an
extensive admin interface, that provides functionality for adding,
removing and changing parts of the system. The admin interface consists
of everything we as developers want the admins to be able to change.
For a complete listing, see figure X.X[kap 2]. We decided to use
Grappelli, an app for the django admin interface that also provided us
with more adequate functionality, e.g.\ auto-completion, rich text
editors, drag {\textquoteleft}n drop and more.

The structure of the layout is simple. Each category has
it's own header and everything in blue is clickable.
The ``Recent Actions'' box is there
to help admins remember what they last did, which is important to
reduce the users short-term memory load, in accordance with
Shneiderman's eigh\t rule.

Originally all the names of the elements were the same as our model
names. We decided to change this to more intuitively understandable
expressions after a request from the customer. Django's
admin interface couldn't give us all the functionality
we wanted, so we had to extend the interface with out own custom views.
We created two views, ``Balloon
overview'' and ``Judge
overview''. To avoid having to create a similar
interface as the rest of admin site, just with different functionality,
we decided to extend the interface templates used for the django admin
interface. This allowed us to change what we wanted, while it still
kept its consistency with the other parts of the admin site.



 \begin{figure}[h!]
	\includegraphics[width=6.5in,heigh\t=3.0417in]{./src/figure/a07Design-img9.png} 
	\caption{FIGURECAPTION}
\end{figure}
fig 7.7

The judge overview was made primarily for judges, but could also be used
by the admins. The motivation behind making this view, is that it gives
the judges an easier overlook over the competition and how the progress
is going for the different teams. We were initially told that the
judges wanted a way to see if a team was struggling, so they could help
that team.

The view consists of four different tables, with the same layout as the
balloon tables. The first two tables depicts how many failed attempts
an onsite or offsite team has. The Problem Overview table provides
statistics on each problem for the gi\ven contest. This was added so
that the judges can see which problem has the most failed or successful
attempts, and if necessary make changes. To make it easy for the judges
to choose a specific team, independent of submissions, we made a
dropdown menu with all the teams. The last table is the highscore list.
We wanted everything to be on one page for the judges, so they
wouldn't have to constantly switch between different
pages.

 \begin{figure}[h!]
	\includegraphics[width=6.5in,heigh\t=2.8752in]{./src/figure/a07Design-img10.png} 
	\caption{FIGURECAPTION}
\end{figure}
Figure \ [Judge\_overview for Team]

Figure \ [Judge\_overview for Team] shows the judge overview after
selecting the team ``GentleCoding''.
It is possible to expand each submission by clicking on it. The third
submission has been clicked on, so we can now choose to expand
different categories. For example if a judge wants to see the source
code for that submission, he/she can click on ``Source
code'' and it will expand. Submissions that
haven't been compiled are shown in red, and the other
are white.

\href{http://www.clevelandconsultinggroup.com/articles/emerge\nce-ge\stalt-approach-to-change.php}
\href{https://www.cs.umd.edu/users/ben/goldenrules.html}
